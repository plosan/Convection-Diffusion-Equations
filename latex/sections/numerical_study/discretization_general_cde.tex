
\subsection{Discretization of the general convection--diffusion equation}

The generalized convection--diffusion for a real valued function $\phi \colon
\Omega \times I \subset \real^n \times \real \rightarrow \real$ is
\begin{equation} \label{eq:discretization_cde_equation_1}
	\pdv{(\rho \phi)}{t} + \div(\rho \vb{v} \phi) = 
	\div(\Gamma_\phi \grad{\phi}) + \dot{s}_\phi,
	\quad (x, t) \in \Omega \times I
\end{equation}
whereas for a vector valued function $\phi = (\phi_1, \ldots, \phi_n) \colon
\Omega \times I \subset \real^n \times \real \rightarrow \real^n$ it is written
as
\begin{equation}
	\pdv{(\rho \phi)}{t} + \div(\rho \vb{v} \otimes \phi) = 
	\div(\Gamma_\phi \grad{\phi}) + \dot{s}_\phi,
	\quad (x, t) \in \Omega \times I
\end{equation}
where $\otimes$ denotes the outer product of $\vb{v} \colon \Omega \times I
\subset \real^n \times \real \rightarrow \real^n$ and $\phi$, which is a $n
\times n$ matrix. Since the generalized convection--diffusion equation for a
vector valued function actually comprises $n$ equations, one for each component
function, we will only study the discretization for a real valued function.

Integrating \eqref{eq:discretization_cde_equation_1} over $\cv{P} \times [t^n,
t^{n+1}] \subset \Omega \times I$ and using Fubini's theorem
(\colorbox{red}{Fubini's theorem}) to swap the order of integration
\begin{multline}
	\int_{t^n}^{t^{n+1}} \int_{\cv{P}} \pdv{(\rho \phi)}{t} \dd{x} \dd{t} + 
	\int_{t^n}^{t^{n+1}} \int_{\cv{P}} \div(\rho \vb{v} \phi) \dd{x} \dd{t} = \\ = 
	\int_{t^n}^{t^{n+1}} \int_{\cv{P}} \div(\Gamma_\phi \grad{\phi}) \dd{x} \dd{t} +
	\int_{t^n}^{t^{n+1}} \int_{\cv{P}} \dot{s}_\phi \dd{x} \dd{t}	
\end{multline}
The simplification of the first term is analogous to that of the continuity
equation. The average value of $\rho \phi$ on $\cv{P}$ at time $t$ is defined by
\begin{equation}
	(\rho \phi)_P = \frac{1}{V_P} \int_{\cv{P}} \rho \phi \dd{x}
\end{equation}
although the following approximation is needed:
\begin{equation}
	(\rho \phi)_P \approx \rho_P \phi_P
\end{equation}
Then the transient term is:
\begin{equation}
	\int_{t^n}^{t^{n+1}} \int_{\cv{P}} \pdv{(\rho \phi)}{t} \dd{x} \dd{t} = 
	\int_{t^n}^{t^{n+1}} \frac{\dd}{\dd{t}} \int_{\cv{P}} \rho \phi \dd{x} \dd{t} =  
	\int_{t^n}^{t^{n+1}} \frac{\dd(\rho \phi)_P}{\dd{t}} V_P \dd{t} \approx 
	\left\{ \rho_P \phi_P - \rho_P^0 \phi_P^0 \right\} V_P
\end{equation}

Divergence theorem must be applied to simplify the convective term,
\begin{equation} 
	\int_{t^n}^{t^{n+1}} \int_{\cv{P}} \div(\rho \vb{v} \phi) \dd{x} \dd{t} = 
	\int_{t^n}^{t^{n+1}} \int_{\cs{P}} \rho \phi \vb{v} \vdot \vb{n} \dd{S} \dd{t} = 
	\int_{t^n}^{t^{n+1}} \sum_i \int_{\cs{Pi}} \rho \phi \vb{v} \vdot \vb{n} \dd{S} \dd{t}
\end{equation}
The value that $\phi$ takes on $\cs{Pi}$ can be approximated by its value at a
representative point, for instance, the point at face center, that is to say,
$\phi \approx \phi_i$. Therefore,
\begin{multline}
	\int_{t^n}^{t^{n+1}} \sum_i \int_{\mathcal{S}_i} \rho \phi \vb{v} \vdot \vb{n} \dd{S} \dd{t} \approx 
	\int_{t^n}^{t^{n+1}} \sum_i \int_{\mathcal{S}_i} \rho \phi_i \vb{v} \vdot \vb{n} \dd{S} \dd{t} =
	\int_{t^n}^{t^{n+1}} \sum_i \dot{m}_i \phi_i \dd{t} = \\
	= \left\{ \beta \sum_i \dot{m}_i \phi_i + (1 - \beta) \sum_i \dot{m}_i^0 \phi_i^0 \right\} \Delta t	
\end{multline}

In regard to the diffusion term,
\begin{equation} \label{eq:discretization_cde_equation_2}
	\int_{t^n}^{t^{n+1}} \int_{\cv{P}} \div(\Gamma_\phi \grad{\phi}) \dd{x} \dd{t} = 
	\int_{t^n}^{t^{n+1}} \int_{\cs{P}} \Gamma_\phi \grad{\phi} \vdot \vb{n} \dd{S} \dd{t} = 
	\int_{t^n}^{t^{n+1}} \sum_i \int_{\cs{Pi}} \Gamma_\phi \grad{\phi} \vdot \vb{n} \dd{S} \dd{t}
\end{equation}
The outer normal vector to the face $\mathcal{S}_{Pi}$ is constant and points in
the direction of some coordinate axis, hence the dot product $\grad{\phi} \vdot
\vb{n}$ in the face $\cs{Pi}$ equals the partial derivative with respect to
$x_i$ times $\pm 1$, depending on the direction of $\vb{n}$. For east, north and
top faces the sign is positive, whilst for west, south and bottom faces the sign
is negative. Again, $\Gamma_\phi$ will be approximated by the value at the face
center, and partial derivatives will be approximated by a finite centered
difference. In order to simplify the notation, we shall drop the subindex $\phi$ in the diffusion
coefficient $\Gamma_\phi$ and define the coefficients
\begin{align}
	D_f &= \frac{\Gamma_f S_f}{d_{PF}} \\
	D_f^0 &= \frac{\Gamma_f^0 S_f}{d_{PF}}
\end{align}
where $f$ and $F$ refer to the face and to the node, respectively. For a 2D--mesh, equation
\eqref{eq:discretization_cde_equation_2} results in
\begin{align}
	&\int_{t^n}^{t^{n+1}} \sum_i \int_{\cs{Pi}} \Gamma_\phi \grad{\phi} \vdot \vb{n} \dd{S} \dd{t} 
	\approx \nonumber \\ 
	&\approx 
	\int_{t^n}^{t^{n+1}}
	\Big\{ 
	D_e (\phi_E - \phi_P) - D_w (\phi_P - \phi_W) + D_n (\phi_N - \phi_P) - D_s (\phi_P - \phi_S) 
	\Big\} \dd{t} \approx \nonumber \\
	&\approx
	\beta 
	\left\{ 
		D_e (\phi_E - \phi_P) - D_w (\phi_P - \phi_W) + D_n (\phi_N - \phi_P) - D_s (\phi_P - \phi_S) 
	\right\} \Delta t + \nonumber \\
	&+ (1 - \beta)
	\Big\{ 
		D_e^0 (\phi_E - \phi_P) - D_w^0 (\phi_P - \phi_W) + D_n^0 (\phi_N - \phi_P) - D_s^0 (\phi_P - \phi_S) 
	\Big\} \Delta t
\end{align}
In the case of a 3D--mesh, the contributions of top and bottom faces must be
accounted for.

So as to discretize the source term, the mean value of the source function in
$\cv{P}$ at time $t$ is given by
\begin{equation}
	\overline{\dot{s}}_\phi = 
	\frac{1}{V_P} \int_{\cv{P}} \dot{s}_\phi \dd{x}
\end{equation}
If the value of $s_\phi$ is known, the relation $\overline{\dot{s}}_\phi =
\dot{\overline{s}}_\phi$ is true. Indeed, applying differentiation under the
integral sign (Theorem \ref{theo:differentiation_under_the_integral_sign})
\begin{equation}
	\dot{\overline{s}}_\phi = 
	\frac{\dd}{\dd{t}} \overline{s}_\phi = 
	\frac{1}{V_P} \frac{\dd}{\dd{t}} \int_{\cv{P}} s_\phi \dd{x} = 
	\frac{1}{V_P} \int_{\cv{P}} \dot{s}_\phi \dd{x} = 
	\overline{\dot{s}}_\phi
\end{equation}
In most cases, the dependence of $\dot{\overline{s}}_\phi$ on $\phi$ is
complicated. Since the equations obtained until now are linear, the relation
between the source term and the variable would ideally be linear. This linearity
is imposed as follows
\begin{equation}
	\dot{\overline{s}}_\phi = S_C^\phi + S_P^\phi \phi_P
\end{equation}
where the values of $S_C^\phi$ and $S_P^\phi$ may vary with $\phi$
\cite{patankar2008numerical}. Making use of these relations, the source term
integral is discretized as
\begin{equation}
	\int_{t^n}^{t^{n+1}} \int_{\cv{P}} \dot{s}_\phi \dd{x} \dd{t} = 
	\int_{t^n}^{t^{n+1}} 
	\dot{\overline{s}}_{\phi P} V_P \Delta t = 
	\left( S_C^\phi + S_P^\phi \phi_P \right) V_P \Delta t
\end{equation}
As we shall discuss later, the term $S_P^\phi$ must be non--positive.

The discretization of the 2D generalized convection--diffusion equation is
\begin{align}
	&\frac{\rho_P \phi_P - \rho_P^0 \phi_P^0}{\Delta t} V_P + \nonumber \\
	&+ 
	\beta 
	\Big( \dot{m}_e \phi_e - \dot{m}_w \phi_w + \dot{m}_n \phi_n - \dot{m}_s \phi_s \Big) + 
	(1 - \beta) 
	\left( \dot{m}_e^0 \phi_e^0 - \dot{m}_w^0 \phi_w^0 + \dot{m}_n^0 \phi_n^0 - \dot{m}_s^0 \phi_s^0 \right) =  \nonumber \\
	&= 
	\beta 
	\Big\{ 
	D_e (\phi_E - \phi_P) - D_w (\phi_P - \phi_W) + D_n (\phi_N - \phi_P) - D_s (\phi_P - \phi_S) 
	\Big\} + \nonumber \\
	&+ (1 - \beta)
	\Big\{ 
	D_e^0 (\phi_E^0 - \phi_P^0) - D_w^0 (\phi_P^0 - \phi_W^0) + 
	D_n^0 (\phi_N^0 - \phi_P^0) - D_s^0 (\phi_P^0 - \phi_S^0)
	\Big\} + \nonumber \\
	&+ \left( S_C^\phi + S_P^\phi \phi_P \right) V_P \label{eq:general_2d_cde}
\end{align}
In the case of a implicit integration scheme, \ie $\beta = 1$, equation
\eqref{eq:general_2d_cde} is simplified to:
\begin{multline} \label{eq:general_cde_discretized_implicit}
	\frac{\rho_P \phi_P - \rho_P^0 \phi_P^0}{\Delta t} V_P + 
	\dot{m}_e \phi_e - \dot{m}_w \phi_w + \dot{m}_n \phi_n - \dot{m}_s \phi_s = \\ = 
	D_e (\phi_E - \phi_P) - D_w (\phi_P - \phi_W) + D_n (\phi_N - \phi_P) - D_s (\phi_P - \phi_S) +
	\big( S_C^\phi + S_P^\phi \phi_P \big) V_P
\end{multline}
An equivalent and more useful form of the discretization equation can be found
by multiplying \eqref{eq:continuity_equation_2d_discretized} times $\phi_P$ and
subtracting it from \eqref{eq:general_cde_discretized_implicit}, which results
in
\begin{multline} \label{eq:general_cde_discretized_implicit_useful}
	\rho_P^0 \frac{\phi_P - \phi_P^0}{\Delta t} V_P
	+ \dot{m}_e (\phi_e - \phi_P) - \dot{m}_w (\phi_w - \phi_P) 
	+ \dot{m}_n (\phi_n - \phi_P) - \dot{m}_s (\phi_s - \phi_P) 
	= \\
	= D_e (\phi_E - \phi_P) - D_w (\phi_P - \phi_W)
	+ D_n (\phi_N - \phi_P) - D_s (\phi_P - \phi_S)
	+ (S_C^\phi + S_P^\phi \phi_P) V_P
\end{multline}
The 3D analog of \eqref{eq:general_cde_discretized_implicit_useful} includes the
top and bottom faces contributions:
\begin{align} \label{eq:general_cde_discretized_implicit_useful_3d}
	\rho_P^0 \frac{\phi_P - \phi_P^0}{\Delta t} V_P
	&+ \dot{m}_e (\phi_e - \phi_P) - \dot{m}_w (\phi_w - \phi_P)  + \dot{m}_n (\phi_n - \phi_P) 
	\nonumber \\
	&- \dot{m}_s (\phi_s - \phi_P) + \dot{m}_t (\phi_t - \phi_P) - \dot{m}_b (\phi_b - \phi_P) 
	\nonumber \\
	&= D_e (\phi_E - \phi_P) - D_w (\phi_P - \phi_W) + D_n (\phi_N - \phi_P) 
	\nonumber \\
	&- D_s (\phi_P - \phi_S) + D_t (\phi_T - \phi_P) - D_b (\phi_P - \phi_B) + (S_C^\phi + S_P^\phi \phi_P) V_P
\end{align}

