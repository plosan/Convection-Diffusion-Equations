\subsubsection{Central--Difference Scheme (CDS)}

The Central--Difference Scheme assumes a linear distribution for $\phi$ as
illustrated in figure \ref{fig:central_difference_scheme}. 
\begin{figure}[ht]
	\centering
	\begin{tikzpicture}
		% Ground
		\draw[thick] (0,0) -- ++(6,0);
		% Point P
		\filldraw[black] (0.5,0) circle (2pt); \draw[dashed] (0.5,0) --
		++(0,1.5); \node[black, yshift=-0.5cm] at (0.5,0) {$P$};
		\filldraw[sblue] (0.5,1.5) circle (2pt); \node[blue,
		yshift=0.5cm] at (0.5,1.5) {$\phi_P$};
		% Point e
		\filldraw[black] (2.5,0) circle (2pt); \draw[dashed] (2.5,0) --
		++(0,{1.5+3/5}); \node[black, yshift=-0.5cm] at (2.5,0) {$e$};
		\filldraw[sblue] (2.5,{1.5+3/5}) circle (2pt); \node[blue,
		yshift=0.5cm] at (2.5,{1.5+3/5}) {$\phi_e$}; 
		% Point E
		\filldraw[black] (5.5,0) circle (2pt); \draw[dashed] (5.5,0) -- ++(0,3);
		\node[black, yshift=-0.5cm] at (5.5,0) {$E$}; \filldraw[sblue]
		(5.5,3) circle (2pt); \node[blue, yshift=0.5cm] at (5.5,3.0) {$\phi_e$};
		
		% Blue line
		\draw[thick, sblue] (0,{1.5-1.5/10}) -- (6,{1.5+1.5*5.5/5});
	\end{tikzpicture}
	\caption{Central Difference Scheme (CDS).}
	\label{fig:central_difference_scheme}
\end{figure}

\noindent
Thereby $\phi_e$ can be obtained interpolating between $\phi_P$ and $\phi_E$,
\begin{equation} \label{eq:cds_e}
	\phi_e - \phi_P = \frac{d_{Pe}}{d_{PE}} (\phi_E - \phi_P)
\end{equation}
as well as the remaining faces values,
\begin{align}
	\phi_w - \phi_P &= \frac{d_{Pw}}{d_{PW}} (\phi_W - \phi_P) \\
	\phi_n - \phi_P &= \frac{d_{Pn}}{d_{PN}} (\phi_N - \phi_P) \\
	\phi_s - \phi_P &= \frac{d_{Ps}}{d_{PS}} (\phi_S - \phi_P) \label{eq:cds_s}
\end{align}
This yields a $2^{\text{nd}}$ order approximation for $\phi_e$ if $d_{Pe} =
d_{Ee}$. In effect, applying Taylor's theorem about point $x_e$,
\begin{equation} \label{eq:cds_taylor_expansion}
	\phi_P = 
	\phi_e 
	- \left(\pdv{\phi}{x}\right)_e d_{Pe} 
	+ \frac{1}{2} \left(\pdv[2]{\phi}{x}\right)_e d_{Pe}^2 
	+ \frac{1}{6} \left(\pdv[3]{\phi}{x}\right)_{\xi_1} d_{Pe}^3
\end{equation}
The $2^\text{nd}$ order approximation of $(\partial_x \phi)_e$ is given by
\begin{equation} \label{eq:cds_derivative_approximation}
	\left(\pdv{\phi}{x}\right)_e = 
	\frac{\phi_E - \phi_P}{d_{PE}} - \left(\pdv[3]{\phi}{x}\right)_{\xi_2} \frac{d_{PE}^2}{3!} = 	
	\frac{\phi_E - \phi_P}{d_{PE}} - \left(\pdv[3]{\phi}{x}\right)_{\xi_2} \frac{(d_{Pe} + d_{Ee})^2}{3!}
\end{equation}
Introducing \eqref{eq:cds_derivative_approximation} in
\eqref{eq:cds_taylor_expansion} and imposing $d_{Pe} = d_{Ee}$, 
\begin{equation} \label{eq:cds_error_terms}
	\phi_e - \phi_P = 
	\frac{d_{Pe}}{d_{PE}} (\phi_E - \phi_P) - 
	\left( \pdv[2]{\phi}{x} \right)_e \frac{d_{Pe}^2}{2} -
	\left\{ 
	\left( \pdv[3]{\phi}{x} \right)_{\xi_1} + 4 \left( \pdv[3]{\phi}{x} \right)_{\xi_2}
	\right\} 
	\frac{d_{Pe}^3}{6}
\end{equation}
As CDS retains the first term on the left--hand side of
\eqref{eq:cds_error_terms}, the highest order term of the error is $\frac{1}{2}
(\partial_x^2 \phi)_e d_{Pe}^2$, proving that CDS provides a $2^\text{nd}$ order
approximation of $\phi_e$ when $d_{Pe} = d_{Ee}$. Nonetheless, this scheme is
prone to stability problems producing oscillatory outputs since the
approximation is of order higher than $1$.