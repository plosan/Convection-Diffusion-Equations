\subsubsection{Second--order Upwind Linear Extrapolation (SUDS)}

As previously mentioned, incompressible flows and fluids at low Mach number are
more influenced by upstream condition than by downstream conditions. In order to
account for this fact and to ease the study, we introduce a new notation. When
located at the face separating two control volumes, $f$ refers to the face, $D$
is the downstream node, $C$ is the first upstream node and $U$ is the most
upstream node. Some books may use $U$ and $UU$ instead of $C$ and $U$,
respectively.

The Second--order Upwind Linear Extrapolation scheme takes profit of this idea
since it extrapolates $\phi_e$ using a straight line between the values of
$\phi$ at nodes $C$ and $U$. The two possible situations are pictured in figures
\ref{fig:suds_1} and \ref{fig:suds_2}.
\FloatBarrier

\begin{figure}[ht]
	\centering
	\begin{minipage}{.5\textwidth}
		\centering
		\begin{tikzpicture}
			% Points
			\def\zerox{0.5} \def\zeroy{1.5} \def\onex{3} \def\oney{2}
			\def\twox{5.5} \def\twoy{3} \def\coefa{1.5} \def\coefb{0.2}
			\def\coefc{0.04}
			% Ground
			\draw[thick] (0,0) -- ++(6,0);
			% Point W
			\filldraw[black] (\zerox,0) circle (2pt); \draw[dashed] (\zerox,0)
			-- ++(0,\zeroy); \node[black, yshift=-0.5cm] at (\zerox,0) {$W$};
			\node[black, yshift=-1cm] at (\zerox,0) {$(U)$};
			\filldraw[sblue] (\zerox,\zeroy) circle (2pt); \node[blue,
			yshift=0.5cm] at (\zerox,\zeroy) {$\phi_W$};
			% Point P
			\filldraw[black] (\onex,0) circle (2pt); \draw[dashed] (\onex,0) --
			++(0,\oney); \node[black, yshift=-0.5cm] at (\onex,0) {$P$};
			\node[black, yshift=-1cm] at (\onex,0) {$(C)$};
			\filldraw[sblue] (\onex,\oney) circle (2pt); \node[blue,
			yshift=0.5cm] at (\onex,\oney) {$\phi_P$};
			% Point e
			\filldraw[black] ({\onex+1},0) circle (2pt); \draw[dashed]
			({\onex+1},0) -- ++(0,{1.5+0.5*3.5/2.5}); \node[black,
			yshift=-0.5cm] at ({\onex+1},0) {$e$}; \filldraw[sblue]
			({\onex+1},{1.5+0.5*3.5/2.5}) circle (2pt); \node[blue,
			yshift=0.5cm] at ({\onex+1},{1.5+0.5*3.5/2.5}) {$\phi_e$};
			% Point E
			\filldraw[black] (\twox,0) circle (2pt); \draw[dashed] (\twox,0) --
			++(0,\twoy); \node[black, yshift=-0.5cm] at (\twox,0) {$E$};
			\node[black, yshift=-1cm] at (\twox,0) {$(D)$};
			\filldraw[sblue] (\twox,\twoy) circle (2pt); \node[blue,
			yshift=0.5cm] at (\twox,\twoy) {$\phi_E$}; 
			% Blue line
			\draw[scale=1, domain=0:4.5, smooth, variable=\x, sblue,
			thick] plot ({\x}, {1.5+0.2*(\x-0.5)});
			% Mass flow
			\draw[-latex, red, thick] (3.25,0.75) -- ++(1.5,0)
			node[above]{$\dot{m}_e > 0$};
		\end{tikzpicture}
		\captionsetup{width=0.9\textwidth}
		\caption{SUDS when $(\vb{v} \vdot \vb{n})_e > 0$.}
		\label{fig:suds_1}
	\end{minipage}%
	\begin{minipage}{.5\textwidth}
		\centering
		\begin{tikzpicture}
			% Points
			\def\zerox{0.5} \def\zeroy{1.5} \def\onex{3} \def\oney{2.5}
			\def\twox{5.5} \def\twoy{3} \def\coefa{1.5} \def\coefb{0.4}
			\def\coefc{-0.04}
			% Ground
			\draw[thick] (0,0) -- ++(6,0);
			% Point P
			\filldraw[black] (\zerox,0) circle (2pt); \draw[dashed] (\zerox,0)
			-- ++(0,\zeroy); \node[black, yshift=-0.5cm] at (\zerox,0) {$P$};
			\node[black, yshift=-1cm] at (\zerox,0) {$(D)$};
			\filldraw[sblue] (\zerox,\zeroy) circle (2pt); \node[blue,
			yshift=0.5cm] at (\zerox,\zeroy) {$\phi_P$};
			% Point e
			\filldraw[black] ({\zerox+1},0) circle (2pt); \draw[dashed]
			({\zerox+1},0) -- ++(0,{2.5-0.2*1.5}); \node[black, yshift=-0.5cm]
			at ({\zerox+1},0) {$e$}; \filldraw[sblue]
			({\zerox+1},{2.5-0.2*1.5}) circle (2pt); \node[blue, yshift=0.5cm]
			at ({\zerox+1},{2.5-0.2*1.5}) {$\phi_e$};
			% Point e
			\filldraw[black] (\onex,0) circle (2pt); \draw[dashed] (\onex,0) --
			++(0,\oney); \node[black, yshift=-0.5cm] at (\onex,0) {$E$};
			\node[black, yshift=-1cm] at (\onex,0) {$(C)$};
			\filldraw[sblue] (\onex,\oney) circle (2pt); \node[blue,
			yshift=0.5cm] at (\onex,\oney) {$\phi_E$}; 
			% Point ee
			\filldraw[black] (\twox,0) circle (2pt); \draw[dashed] (\twox,0) --
			++(0,\twoy); \node[black, yshift=-0.5cm] at (\twox,0) {$EE$};
			\node[black, yshift=-1cm] at (\twox,0) {$(U)$};
			\filldraw[sblue] (\twox,\twoy) circle (2pt); \node[blue,
			yshift=0.5cm] at (\twox,\twoy) {$\phi_{EE}$}; 
			% Blue line
			\draw[scale=1, domain=1:6, smooth, variable=\x, sblue,
			thick] plot ({\x}, {2.5+0.2*(\x-3)});
			% Mass flow
			\draw[-latex, red, thick] (0.75,0.75) -- ++(1.5,0)
			node[above]{$\dot{m}_e < 0$};
		\end{tikzpicture}
		\captionsetup{width=0.9\textwidth}
		\caption{SUDS when $(\vb{v} \vdot \vb{n})_e < 0$.}
		\label{fig:suds_2}
	\end{minipage}
\end{figure}

\noindent
On the one hand, when $(\vb{v} \vdot \vb{n})_e > 0$, the line between points
$(x_W, \phi_W)$ and $(x_P, \phi_P)$ is given by
\begin{equation}
	\phi(x) = \phi_W + \frac{\phi_P - \phi_W}{d_{PW}} (x - x_W)
\end{equation}
and substituting at $x = x_e$, the formula for $\phi_e$ is obtained:
\begin{equation} \label{eq:suds_1_1}
	\phi_e =
	\phi_W + \frac{\phi_P - \phi_W}{d_{PW}} (x_e - x_W) = 
	\phi_P + \frac{d_{Pe}}{d_{PW}} (\phi_P - \phi_W) 
\end{equation}
On the other hand, in the case of $(\vb{v} \vdot \vb{n})_e < 0$, the line
between points $(x_E, \phi_E)$ and $(x_{EE}, \phi_{EE})$ is
\begin{equation}
	\phi(x) = \phi_E + \frac{\phi_{EE} - \phi_E}{d_{E,EE}} (x - x_E)
\end{equation}
and the approximation of $\phi_e$ is
\begin{equation} \label{eq:suds_2_1}
	\phi_e =
	\phi_E + \frac{\phi_{EE} - \phi_E}{d_{E,EE}} (x_e - x_E) =
	\phi_E + \frac{d_{Ee}}{d_{E,EE}} (\phi_E - \phi_{EE})	
\end{equation}
Using the DCU notation, \eqref{eq:suds_1_1} and \eqref{eq:suds_2_1} are both
rewritten in the following manner:
\begin{equation}
	\phi_f - \phi_C = \frac{d_{Cf}}{d_{CU}} (\phi_C - \phi_U)
\end{equation}

In order to prove that SUDS is a second order scheme when a uniform mesh
is used and $(\vb{v} \vdot \vb{n})_e > 0$, consider the Taylor expansion up to
$2^{nd}$ degree of $\phi$ about point $x_W$,
\begin{equation}
	\phi_e = 
	\phi_W + 
	\left( \pdv{\phi}{x} \right)_W d_{We} + 
	\left( \pdv[2]{\phi}{x} \right)_{\xi_1} \frac{d_{We}^2}{2}
\end{equation}
The first derivative of $\phi$ with respect to $x$ can be replaced by its first
order approximation, namely,
\begin{equation}
	\left(\pdv{\phi}{x}\right)_W = 
	\frac{\phi_P - \phi_W}{d_{PW}} - \left(\pdv[2]{\phi}{x}\right)_{\xi_2} \frac{d_{PW}}{2}
\end{equation}
thereby,
\begin{align}
	\phi_e 
	&= 
	\phi_W + 
	\frac{d_{We}}{d_{PW}} (\phi_P - \phi_W) + 
	\left( \pdv[2]{\phi}{x} \right)_{\xi_1} \frac{d_{We}^2}{2} - 
	\left( \pdv[2]{\phi}{x} \right)_{\xi_2} \frac{d_{We} d_{PW}}{2} \nonumber \\
	&= 
	\phi_P + 
	\frac{d_{Pe}}{d_{PW}} (\phi_P - \phi_W) + 
	\left( \pdv[2]{\phi}{x} \right)_{\xi_1} \frac{(d_{PW} + d_{Pe})^2}{2} - 
	\left( \pdv[2]{\phi}{x} \right)_{\xi_2} \frac{(d_{PW} + d_{Pe}) d_{PW}}{2}	
	\label{eq:suds_error}
\end{align}
The scheme retains the two first terms on the right--hand side of
\eqref{eq:suds_error}, therefore the error is composed by the last two terms.
The uniform mesh hypothesis implies $d_{PW} = 2 d_{Pe} = L$, therefore the error
term is multiplied by $L^2$,
\begin{equation}
	\phi_e = 
	\phi_P + \frac{d_{Pe}}{d_{PW}} (\phi_P - \phi_W) + 
	\frac{3 L^2}{4}
	\left\{
	3 \left( \pdv[2]{\phi}{x} \right)_{\xi_1} - \left( \pdv[2]{\phi}{x} \right)_{\xi_2}
	\right\}
\end{equation}
whence the second order of SUDS is deduced. The proof in the case of $(\vb{v}
\vdot \vb{n})_e < 0$ is analogous.