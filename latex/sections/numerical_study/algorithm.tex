
\subsection{Solving algorithm}

The procedure to solve of a transient convection--diffusion problem with 2D--cartesian mesh is shown in Algorithm \ref{algorithm:general_solving_algorithm}.

\begin{algorithm}[h]
	\caption{Resolution of a transient convection--diffusion problem with 2D--cartesian mesh.}
	\label{algorithm:general_solving_algorithm}
	\begin{algorithmic}[0]
		\State 
		\begin{enumerate}[label=\textbf{\arabic*},topsep=0pt]
			\item Input data:
			\begin{enumerate}[label=\textbf{1.\arabic*}]
				\item Physical data: geometry, thermophysical properties, initial and boundary conditions.
				\item Numerical data: mesh, $\Delta t$ (time step), $\delta$ (convergence criterion).
			\end{enumerate}
			\item Mesh generation: nodes position, faces position, distances, surfaces and volumes. 
			\item Initial map: $n \gets 0$, $t^n \gets 0$, $\phi^0[i][j] = \phi(x,y,t) \rvert_{t = 0}$.
			\item Compute the new time step: $t^{n+1} = t^n + \Delta t$. \label{alg:algorithm2_compute_new_time_step}
			\begin{enumerate}[label=\textbf{4.\arabic*}]
				\item Initial estimated values: $\phi^\ast [i][j] \gets \phi^0[i][j]$.
				\item Evaluation of the discretization coefficients: $a_E[i][j]$, $a_W[i][j]$, $a_N[i][j]$, $a_S[i][j]$, $a_P[i][j]$, $b_P[i][j]$. \label{alg:algorithm2_discretization_coefficients}
				\item Resolution of the linear system \label{alg:algorithm2_linear_system}
				\begin{align*}
					a_P[i][j] \, \phi[i][j] 
					= a_E[i][j] \, \phi[i+1][j] 
					&+ a_W[i][j] \, \phi[i-1][j] \\
					&+ a_N[i][j] \, \phi[i][j+1]
					+ a_S[i][j] \, \phi[i][j-1] 
					+ b_P[i][j]
				\end{align*}
				\item Is $\displaystyle \max_{i,j} \abs{\phi^\ast[i][j] - \phi[i][j]} < \delta$? \label{alg:algorithm2_convergence_condition}
				\begin{itemize}
					\item Yes: continue.
					\item No: $\phi^\ast[i][j] \gets \phi[i][j]$, go to \ref{alg:algorithm2_discretization_coefficients}.
				\end{itemize}
			\end{enumerate}
			\item New time step?
			\begin{itemize}
				\item Yes: $n \gets n + 1$, go to \ref{alg:algorithm2_compute_new_time_step}.
				\item No: continue.
			\end{itemize}
			\item Final computations, print results.
			\item End.
		\end{enumerate}
	\end{algorithmic}
\end{algorithm}

Hereinafter, the term iteration will be used to refer to the iterative procedure to solve the linear system on step \ref{alg:algorithm2_linear_system}. It must not be confused with the next time instant term.

On step \ref{alg:algorithm2_compute_new_time_step} the next time instant is computed. This is the most computationally expensive step in the algorithm, specially part \ref{alg:algorithm2_linear_system} where the resolution of the linear system of discretized equations is carried out. As a result of the convection--diffusion equations nature, the system matrix $A$ and the vector of independent terms $b$ change each time the convergence condition is not fulfilled on step \eqref{alg:algorithm2_convergence_condition}. Since $A$ and $b$ depend on the previous iteration value of $\phi$, that is to say, $A = A(\phi^\ast)$ and $b = b(\phi^\ast)$, the linear system of equations is
\begin{equation} \label{eq:algorithm_linear_system}
	A(\phi^\ast) \, \phi = b(\phi^\ast)
\end{equation}
In the case both $A$ and $b$ were constant, the algorithm needed to solve the linear system \eqref{eq:algorithm_linear_system} would be clear at first glance. Since each node is influenced only by the adjacent nodes, $A$ is a pentadiagonal by blocks matrix, therefore $A$ is sparse, \ie most of the elements are zero, hence an iterative method for solving the linear system would be convenient. Let $A_{ij}$ denote the element in the $i$-th row and $j$-th column of $A$. If $A$ is, in addition, a strictly diagonally dominant (SDD) matrix, that is to say,
\begin{equation} \label{eq:sdd_matrix_1}
	\abs{A_{ii}} \geq 
	\sum_{\substack{j = 1 \\ j \neq i}}^n \abs{A_{ij}} 
\end{equation}
then Gauss--Seidel algorithm is guaranteed to converge and eventually solve the system. In terms of the discretization coefficients, condition \eqref{eq:sdd_matrix_1} is written as
\begin{equation}
	\abs{a_P[i][j]} \geq 
	\abs{a_W[i][j]} +
	\abs{a_E[i][j]} +
	\abs{a_S[i][j]} +
	\abs{a_N[i][j]}
\end{equation}
Nonetheless the SDD condition is not assured as a consequence of the discretization coefficients.

As the execution of an iterative procedure to solve the linear system may diverge, arguably the most convenient is a direct method.

