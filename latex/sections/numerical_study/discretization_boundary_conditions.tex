\subsection{Treatment of boundary conditions}

In Cauchy problems involving Partial Differential Equations (PDEs), there exist
several kinds of boundary conditions which must be prescribed in order to
guarantee the existence and uniqueness of solution, although in this project
we will only consider two types. So as to illustrate how these conditions are set,
let $U \subset \real^n$ be a bounded open subset of $\real^n$. The heat or diffusion
equation is the PDE
\begin{equation} \label{eq:boundary_conditions_heat_equation}
	u_t - \Delta u = f(x, t) \quad (x, t) \in U \times (0,\infty)
\end{equation}
where $\Delta = \sum_{i=1}^{m} \pdv[2]{}{x_i}$ is Laplace's operator and $f$
models the internal sources for magnitude $u$ \cite{evans1998pde}. This equation models the
evolution in time of the density $u$ of some quantity such as heat, chemical
concentration, etc. Let $g \colon U \rightarrow \real$ be the
initial value for $u$. The typical Cauchy problem for diffusion equation is
\begin{equation} \label{eq:boundary_conditions_problem}
	\left\{
	\begin{aligned}
		&u_t - \Delta u = f(x, t) & &\text{in } U \times (0,\infty) \\
		&u = g & &\text{on } U \times \{ t = 0 \} \\
		&\text{Boundary conditions}
	\end{aligned}
	\right.
\end{equation}
The boundary conditions considered are:
\begin{itemize}[topsep=0pt]
	\item Dirichlet boundary condition: the value of $u$ is prescribed on
	$\partial U \times (0,\infty)$, that is to say, if $d \colon \partial U
	\times (0,\infty) \rightarrow \real$, $(x, t) \mapsto d(x, t)$
	describes the boundary condition, then
	\eqref{eq:boundary_conditions_problem} is written as
	\begin{equation}
		\left\{
		\begin{aligned}
			&u_t - \Delta u = f(x, t) & &\text{in } U \times (0,\infty) \\
			&u = g & &\text{on } U \times \{ t = 0 \} \\
			&u = d & &\text{on } \partial U \times (0,\infty)
		\end{aligned}
		\right.
	\end{equation}
	When \eqref{eq:boundary_conditions_heat_equation} is thought of as
	describing the propagation of heat, then $d$ fixes the temperature at the
	boundary of $U$ for each time.
	\item Neumann boundary condition: the normal derivative of $u$ to the
	boundary of $U$ is prescribed on $\partial U \times (0,\infty)$, \ie if $n
	\colon \partial U \times (0,\infty) \rightarrow \real$ describes the
	boundary condition, then \eqref{eq:boundary_conditions_problem} is written
	as
	\begin{equation}
		\left\{
		\begin{aligned}
			&u_t - \Delta u = f(x, t) 	& &\text{in } U \times (0,\infty) \\
			&u = g 							& &\text{on } U \times \{ t = 0 \} \\
			&\partial_{\nu} u = n 			& &\text{on } \partial U \times (0,\infty)
		\end{aligned}
		\right.
	\end{equation}
	where $\nu$ is the outer normal vector to $\partial U$. In terms of heat,
	this boundary condition sets the conduction heat transfer through $U$ for
	each time.
\end{itemize}

The numerical treatment of boundary conditions is straightforward, specially is
our case as we are using a cartesian mesh on a rectangular domain. In the
case of a Dirichlet boundary condition, such as the one shown in figure
\ref{fig:boundary_conditions_dirichlet}, the value at face must be equal to the
prescribed value at boundary, that is,
\begin{equation}
	\phi_e = \phi_E
\end{equation}
and flux per unit of surface can be easily computed as
\begin{equation}
	j_e = -\Gamma_P \frac{\phi_e - \phi_P}{d_{Pe}}
\end{equation}
In contrast, when a Neumann boundary condition with flux $j_e$ is imposed, the value at
face is
\begin{equation}
	\phi_e = \phi_P - \frac{j_e d_{Pe}}{\Gamma_P}
\end{equation}
This second situation is pictured in figure
\ref{fig:boundary_conditions_neumann}.

\begin{figure}\centering
	\begin{minipage}{.5\textwidth}
		\centering
		\begin{tikzpicture}
			\def\nl{1.5cm}
			\def\of{0.25cm}
			\def\ys{3mm}
			% Canvas
			\fill[white] (-\of,{-\nl-\of}) rectangle (3*\nl,{2*\nl+\of});
			% Horizontal lines control volumes
			\draw[dashed] (-\of,-\nl) -- (1.5*\nl,-\nl);
			\draw[dashed] (-\of,0) -- (1.5*\nl,0);
			\draw[dashed] (-\of,\nl) -- (1.5*\nl,\nl);
			\draw[dashed] (-\of,2*\nl) -- (1.5*\nl,2*\nl);
			% Vertical lines control volumes
			\draw[dashed] (0,-\nl-\of) -- (0,2*\nl+\of);
			\draw[dashed] (\nl,-\nl-\of) -- (\nl,2*\nl+\of);
			\draw[thick] (1.5*\nl,-\nl-\of) -- (1.5*\nl,2*\nl+\of);
			% Nodes
			\foreach \x in {0,1} {
				\foreach \y in {-1,0,1} {
					\filldraw[sblue] (\x*\nl+0.5*\nl,\y*\nl+0.5*\nl) circle (2pt);
				}
			}
			% Text
			\node[sblue, yshift=\ys] at (0.5*\nl, 0.5*\nl) {$P$};
			\node[sblue, yshift=\ys] at (0.5*\nl, 1.5*\nl) {$N$};
			\node[sblue, yshift=\ys] at (0.5*\nl, -0.5*\nl) {$S$};
			\node[circle, inner sep=0pt, outer sep=0pt, yshift=\ys, sblue, fill=white] at ({1.5*\nl},0.5*\nl) {$E$};
			\node[sblue, anchor=west, xshift=1mm] at ({1.5*\nl}, 0.5*\nl) {$\phi_e = \phi_E$};
		\end{tikzpicture}
		\captionsetup{width=0.9\textwidth}
		\caption{Dirichlet boundary condition.}
		\label{fig:boundary_conditions_dirichlet}
	\end{minipage}%
	\begin{minipage}{.5\textwidth}
		\centering
		\begin{tikzpicture}
			\def\nl{1.5cm}
			\def\of{0.25cm}
			\def\ys{3mm}
			% Canvas
			\fill[white] (-\of,{-\nl-\of}) rectangle (3*\nl,{2*\nl+\of});
			% Horizontal lines control volumes
			\draw[dashed] (-\of,-\nl) -- (1.5*\nl,-\nl);
			\draw[dashed] (-\of,0) -- (1.5*\nl,0);
			\draw[dashed] (-\of,\nl) -- (1.5*\nl,\nl);
			\draw[dashed] (-\of,2*\nl) -- (1.5*\nl,2*\nl);
			% Vertical lines control volumes
			\draw[dashed] (0,-\nl-\of) -- (0,2*\nl+\of);
			\draw[dashed] (\nl,-\nl-\of) -- (\nl,2*\nl+\of);
			\draw[thick] (1.5*\nl,-\nl-\of) -- (1.5*\nl,2*\nl+\of);
			% Nodes
			\foreach \x in {0,1} {
				\foreach \y in {-1,0,1} {
					\filldraw[sblue] (\x*\nl+0.5*\nl,\y*\nl+0.5*\nl) circle (2pt);
				}
			}
			% Text
			\node[sblue, yshift=\ys] at (0.5*\nl, 0.5*\nl) {$P$};
			\node[sblue, yshift=\ys] at (0.5*\nl, 1.5*\nl) {$N$};
			\node[sblue, yshift=\ys] at (0.5*\nl, -0.5*\nl) {$S$};
			\node[circle, inner sep=0pt, outer sep=0pt, yshift=\ys, sblue, fill=white] at ({1.5*\nl},0.5*\nl) {$E$};			
			% Arrow
			\draw[->, red, thick] (1.05*\nl,0.5*\nl) -- ++(0.9*\nl,0) node[anchor=west]{$j_e = j_E$};
		\end{tikzpicture}
		\captionsetup{width=0.9\textwidth}
		\caption{Neumann boundary condition.}
		\label{fig:boundary_conditions_neumann}
	\end{minipage}
\end{figure}

