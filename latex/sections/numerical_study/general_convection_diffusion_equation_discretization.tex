
\subsection{Discretization of the general convection--diffusion equation}

The generalized convection--diffusion for a real valued function $\phi \colon \Omega \times I \subset \real^m \times \real \rightarrow \real$ is
\begin{equation} \label{eq:discretization_cde_equation_1}
	\pdv{(\rho \phi)}{t} + \div(\rho \vb{v} \phi) = 
	\div(\Gamma_\phi \grad{\phi}) + \dot{s}_\phi,
	\quad (\vb{x}, t) \in \Omega \times I
\end{equation}
whereas for a vector valued function $\phi = (\phi_1, \ldots, \phi_m) \colon \Omega \times I \subset \real^m \times \real \rightarrow \real^m$ it is written as
\begin{equation}
	\pdv{(\rho \phi)}{t} + \div(\rho \vb{v} \otimes \phi) = 
	\div(\Gamma_\phi \grad{\phi}) + \dot{s}_\phi,
	\quad (\vb{x}, t) \in \Omega \times I
\end{equation}
where $\otimes$ denotes the outer product of $\vb{v} \colon \Omega \times I \subset \real^m \times \real \rightarrow \real^m$ and $\phi$, which is a $m \times m$ matrix. Since the generalized convection--diffusion equation for a vector valued function actually comprises $m$ equations, one for each component function, only the discretization for a real valued function will be studied. 

Integrating \eqref{eq:discretization_cde_equation_1} over $\cv{P} \times [t^n, t^{n+1}] \subset \Omega \times I$ and using Fubini's theorem
\begin{multline}
	\int_{t^n}^{t^{n+1}} \int_{\cv{P}} \pdv{(\rho \phi)}{t} \dd{\vb{x}} \dd{t} + 
	\int_{t^n}^{t^{n+1}} \int_{\cv{P}} \div(\rho \vb{v} \phi) \dd{\vb{x}} \dd{t} = \\ = 
	\int_{t^n}^{t^{n+1}} \int_{\cv{P}} \div(\Gamma_\phi \grad{\phi}) \dd{\vb{x}} \dd{t} +
	\int_{t^n}^{t^{n+1}} \int_{\cv{P}} \dot{s}_\phi \dd{\vb{x}} \dd{t}	
\end{multline}
The simplification of the first term is analogous to that of the continuity equation. The average value of $\rho \phi$ on $\cv{P}$ at time $t$ is defined by
\begin{equation}
	(\rho \phi)_P = \frac{1}{V_P} \int_{\cv{P}} \rho \phi \dd{\vb{x}}
\end{equation}
although the following approximation is needed:
\begin{equation}
	(\rho \phi)_P \approx \rho_P \phi_P
\end{equation}
Then the transient term is:
\begin{equation}
	\int_{t^n}^{t^{n+1}} \int_{\cv{P}} \pdv{(\rho \phi)}{t} \dd{\vb{x}} \dd{t} = 
	\int_{t^n}^{t^{n+1}} \frac{\dd}{\dd{t}} \int_{\cv{P}} \rho \phi \dd{\vb{x}} \dd{t} =  
	\int_{t^n}^{t^{n+1}} \frac{\dd(\rho \phi)_P}{\dd{t}} V_P \dd{t} \approx 
	\left\{ \rho_P \phi_P - \rho_P^0 \phi_P^0 \right\} V_P
\end{equation}

Divergence theorem must be applied to simplify the convective contribution,
\begin{equation} 
	\int_{t^n}^{t^{n+1}} \int_{\cv{P}} \div(\rho \vb{v} \phi) \dd{\vb{x}} \dd{t} = 
	\int_{t^n}^{t^{n+1}} \int_{\cs{P}} \rho \phi \vb{v} \vdot \vb{n} \dd{S} \dd{t} = 
	\int_{t^n}^{t^{n+1}} \sum_i \int_{\cs{Pi}} \rho \phi \vb{v} \vdot \vb{n} \dd{S} \dd{t}
\end{equation}
The value that $\phi$ takes on $\cs{Pi}$ can be approximated by its value at a representative point, for instance, the point at face center, that is to say, $\phi \approx \phi_i$. Therefore,
\begin{multline}
	\int_{t^n}^{t^{n+1}} \sum_i \int_{\mathcal{S}_i} \rho \phi \vb{v} \vdot \vb{n} \dd{S} \dd{t} \approx 
	\int_{t^n}^{t^{n+1}} \sum_i \int_{\mathcal{S}_i} \rho \phi_i \vb{v} \vdot \vb{n} \dd{S} \dd{t} =
	\int_{t^n}^{t^{n+1}} \sum_i \dot{m}_i \phi_i \dd{t} = \\
	= \left\{ \beta \sum_i \dot{m}_i \phi_i + (1 - \beta) \sum_i \dot{m}_i^0 \phi_i^0 \right\} \Delta t	
\end{multline}

For the third term,
\begin{equation} \label{eq:discretization_cde_equation_2}
	\int_{t^n}^{t^{n+1}} \int_{\cv{P}} \div(\Gamma_\phi \grad{\phi}) \dd{\vb{x}} \dd{t} = 
	\int_{t^n}^{t^{n+1}} \int_{\cs{P}} \Gamma_\phi \grad{\phi} \vdot \vb{n} \dd{S} \dd{t} = 
	\int_{t^n}^{t^{n+1}} \sum_i \int_{\cs{Pi}} \Gamma_\phi \grad{\phi} \vdot \vb{n} \dd{S} \dd{t}
\end{equation}
Since a cartesian mesh is being used, the outer normal vector to the face $\mathcal{S}_{Pi}$ is constant and points in the direction of some coordinate axis. Hence the dot product $\grad{\phi} \vdot \vb{n}$ in the face $\cs{Pi}$ equals the partial derivative with respect to $x_i$ times $\pm 1$, depending on the direction of $\vb{n}$. For east, north and top faces the sign is positive, whilst for west, south and bottom faces the sign is negative. Again, $\Gamma_\phi$ will be approximated by the value at the face center, and partial derivatives will be approximated by a finite centered difference. In order to simplify the notation, subindex $\phi$ in the diffusion coefficient $\Gamma_\phi$ will be dropped. For short, the following magnitudes are defined
\begin{align}
	D_i &= \frac{\Gamma_i S_i}{d_{PI}} \\
	D_i^0 &= \frac{\Gamma_i^0 S_i}{d_{PI}}
\end{align}
where $i$ and $I$ refer to the face letter. For a 2D--mesh, equation \eqref{eq:discretization_cde_equation_2} results in
\begin{align}
	&\int_{t^n}^{t^{n+1}} \sum_i \int_{\cs{Pi}} \Gamma_\phi \grad{\phi} \vdot \vb{n} \dd{S} \dd{t} 
	\approx \nonumber \\ 
	&\approx 
	\int_{t^n}^{t^{n+1}}
	\Big\{ 
	D_i (\phi_E - \phi_P) - D_w (\phi_P - \phi_W) + D_n (\phi_N - \phi_P) - D_s (\phi_P - \phi_S) 
	\Big\} \dd{t} \approx \nonumber \\
	&\approx
	\beta 
	\left\{ 
		D_e (\phi_E - \phi_P) - D_w (\phi_P - \phi_W) + D_n (\phi_N - \phi_P) - D_s (\phi_P - \phi_S) 
	\right\} \Delta t + \nonumber \\
	&+ (1 - \beta)
	\Big\{ 
		D_e^0 (\phi_E - \phi_P) - D_w^0 (\phi_P - \phi_W) + D_n^0 (\phi_N - \phi_P) - D_s^0 (\phi_P - \phi_S) 
	\Big\} \Delta t
\end{align}
In the case of a 3D--mesh, the contributions of top and bottom faces must be accounted for.

In order to discretize the fourth term, the mean value of the source in $\cv{P}$ at time $t$ is by
\begin{equation}
	\overline{\dot{s}}_\phi = 
	\frac{1}{V_P} \int_{\cv{P}} \dot{s}_\phi \dd{\vb{x}}
\end{equation}
If the value of $s_\phi$ is known, the relation $\overline{\dot{s}}_\phi = \dot{\overline{s}}_\phi$ is true. Indeed,
\begin{equation}
	\dot{\overline{s}}_\phi = 
	\frac{\dd}{\dd{t}} \overline{s}_\phi = 
	\frac{1}{V_P} \frac{\dd}{\dd{t}} \int_{\cv{P}} s_\phi \dd{\vb{x}} = 
	\frac{1}{V_P} \int_{\cv{P}} \dot{s}_\phi \dd{\vb{x}} = 
	\overline{\dot{s}}_\phi
\end{equation}
In most cases, the dependence of $\dot{\overline{s}}_\phi$ on $\phi$ is complicated. Since the equations obtained until now are linear, the relation between the source term and the variable would ideally be linear. This linearity is imposed as follows
\begin{equation}
	\dot{\overline{s}}_\phi = S_C^\phi + S_P^\phi \phi_P
\end{equation}
where the values of $S_C^\phi$ and $S_P^\phi$ may vary with $\phi$ \cite{patankar2008numerical}. Making use of these relations, the source term integral is discretized as
\begin{equation}
	\int_{t^n}^{t^{n+1}} \int_{\cv{P}} \dot{s}_\phi \dd{\vb{x}} \dd{t} = 
	\int_{t^n}^{t^{n+1}} 
	\dot{\overline{s}}_{\phi P} V_P \Delta t = 
	\left( S_C^\phi + S_P^\phi \phi_P \right) V_P \Delta t
\end{equation}
As shall be discussed later, $S_P^\phi$ must be non--positive.

Finally, the discretization of the 2D generalized convection--diffusion equation is
\begin{align}
	&\frac{\rho_P \phi_P - \rho_P^0 \phi_P^0}{\Delta t} V_P + \nonumber \\
	&+ 
	\beta 
	\Big( \dot{m}_e \phi_e - \dot{m}_w \phi_w + \dot{m}_n \phi_n - \dot{m}_s \phi_s \Big) + 
	(1 - \beta) 
	\left( \dot{m}_e^0 \phi_e^0 - \dot{m}_w^0 \phi_w^0 + \dot{m}_n^0 \phi_n^0 - \dot{m}_s^0 \phi_s^0 \right) =  \nonumber \\
	&= 
	\beta 
	\Big\{ 
	D_e (\phi_E - \phi_P) - D_w (\phi_P - \phi_W) + D_n (\phi_N - \phi_P) - D_s (\phi_P - \phi_S) 
	\Big\} + \nonumber \\
	&+ (1 - \beta)
	\Big\{ 
	D_e^0 (\phi_E^0 - \phi_P^0) - D_w^0 (\phi_P^0 - \phi_W^0) + 
	D_n^0 (\phi_N^0 - \phi_P^0) - D_s^0 (\phi_P^0 - \phi_S^0)
	\Big\} + \nonumber \\
	&+ \left( S_C^\phi + S_P^\phi \phi_P \right) V_P \label{eq:general_2d_cde}
\end{align}
In the case of a implicit integration scheme, that is, $\beta = 1$, equation \eqref{eq:general_2d_cde} is simplified to:
\begin{multline}
	\frac{\rho_P \phi_P - \rho_P^0 \phi_P^0}{\Delta t} V_P + 
	\dot{m}_e \phi_e - \dot{m}_w \phi_w + \dot{m}_n \phi_n - \dot{m}_s \phi_s = \\ = 
	D_e (\phi_E - \phi_P) - D_w (\phi_P - \phi_W) + D_n (\phi_N - \phi_P) - D_s (\phi_P - \phi_S) +
	\big( S_C^\phi + S_P^\phi \phi_P \big) V_P
\end{multline}

