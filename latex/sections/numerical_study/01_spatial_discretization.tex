
\subsection{Spatial discretization}

Suppose that the problem we are studying takes place in a bounded domain $\Omega \subset \real^m$ with $1 \leq m \leq 3$ depending on the case. In order to solve the problem numerically, the domain is discretized into nonoverlapping control volumes and a control node is placed at the center of each one \cite{patankar2018numerical}. There exist two manners to discretize the domain, namely, the cell--centered and the node--centered discretization. The former places discretization nodes over the domain and then generates a control volume centered on each node. The latter first generates the control volumes and then places a node at the center of each one.
