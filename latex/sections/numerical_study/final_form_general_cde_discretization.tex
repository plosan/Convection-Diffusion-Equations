\subsection{Final form of the generalized convection--diffusion equation}

The purpose of this subsection is to obtain a discretization equation of the
form
\begin{equation} \label{eq:final_form_1}
	\mathcal{A}_P \phi_P + \sum_F \mathcal{A}_F \phi_F = \mathcal{Q}_P
\end{equation}
so that it can be easily implemented to be solved numerically, starting from
equation \eqref{eq:general_cde_discretized_implicit} and the studied schemes to
evaluate convective properties. Among the revised schemes, some use only the
surrounding nodes, whilst others involve a larger amount of nodes. As a
consequence of the different treatment needed, separate subsections are devoted
to each type of scheme.

\subsubsection{Small molecule schemes}

Small molecule schemes are those which only involve adjacent nodes to the volume
faces. That is, index $I$ in \eqref{eq:final_form_1} refers to nodes $E$, $W$,
$N$ and $S$. As a result, these schemes can be introduced in a compact form as
\colorbox{red}{Patankar} suggests \cite{patankar2008numerical}. However a
different approach will be followed here. 

As it can be noted from the expressions for schemes UDS (equations
\eqref{eq:uds_e} to \eqref{eq:uds_s}), CDS (equations \eqref{eq:cds_e} to
\eqref{eq:cds_s}) and EDS (equations \eqref{eq:eds_e} to \eqref{eq:eds_s}), once
a face $f$ is chosen, the formula for $\phi_f$ has the following form
\begin{equation} \label{eq:small_molecule_schemes_1}
	\phi_f - \phi_P = A_f (\phi_F - \phi_P)
\end{equation}
where $A_f$ is a coefficient which depends on the face and the scheme.
Particularizing for each face,
\begin{align}	
	\phi_e - \phi_P &= A_e (\phi_E - \phi_P) \label{eq:small_molecule_schemes_2} \\ 
	\phi_w - \phi_P &= A_w (\phi_W - \phi_P) \\
	\phi_n - \phi_P &= A_n (\phi_N - \phi_P) \\
	\phi_s - \phi_P &= A_s (\phi_S - \phi_P) \label{eq:small_molecule_schemes_3} 
\end{align}
The values of $A_f$ and $B_f$ for each face and scheme are collected in table
\ref{tab:small_molecule_schemes_coefficients}.

\begin{table}[h]
	\centering
	\renewcommand{\arraystretch}{1.5}
	\begin{tabular}{lccc}
		\toprule[0.5mm]
		Face & UDS & CDS & EDS \\
		\midrule[0.25mm]
		East & $\dfrac{\dot{m}_e - \abs{\dot{m}_e}}{2 \dot{m}_e}$ &
		$\dfrac{d_{Pe}}{d_{PE}}$ & $\dfrac{e^{\mathrm{Pe}_e
		\frac{d_{Pe}}{d_{PE}}} - 1}{e^{\mathrm{Pe}_e} - 1}$,  
		$\mathrm{Pe}_e = \dfrac{\rho_e u_e d_{PE}}{\Gamma_e}$ \\
		
		West & $\dfrac{\dot{m}_w + \abs{\dot{m}_w}}{2 \dot{m}_w}$ &
		$\dfrac{d_{Pw}}{d_{PW}}$ & $\mathrm{Pe}_w = \dfrac{\rho_w u_w
		d_{PW}}{\Gamma_w}$ \\
		
		North & $\dfrac{\dot{m}_n - \abs{\dot{m}_n}}{2 \dot{m}_n}$ &
		$\dfrac{d_{Pn}}{d_{PN}}$ & $\dfrac{e^{\mathrm{Pe}_n
		\frac{d_{Pn}}{d_{PN}}} - 1}{e^{\mathrm{Pe}_n} - 1}$,  
		$\mathrm{Pe}_n = \dfrac{\rho_n v_n d_{PN}}{\Gamma_n}$ \\
		
		South & $\dfrac{\dot{m}_s + \abs{\dot{m}_s}}{2 \dot{m}_s}$ &
		$\dfrac{d_{Ps}}{d_{PS}}$ & $\mathrm{Pe}_s = \dfrac{\rho_s v_s
		d_{PS}}{\Gamma_s}$ \\
		\bottomrule[0.5mm]
	\end{tabular}
	\captionsetup{width=0.7\linewidth}
	\caption{Coefficient $A_f$ of equation $\phi_f - \phi_P = A_f (\phi_F - \phi_P)$ for east, west, north and south faces, and for schemes UDS, CDS and EDS.}
	\label{tab:small_molecule_schemes_coefficients}
\end{table}

\noindent
Introducing equations \eqref{eq:small_molecule_schemes_2} through
\eqref{eq:small_molecule_schemes_3} in
\eqref{eq:general_cde_discretized_implicit_useful}, a general discretization
equation comprising UDS, CDS and EDS can be obtained:
\begin{align} 
	&\left\{
	D_e - \dot{m}_e A_e +
	D_w + \dot{m}_w A_w +
	D_n - \dot{m}_n A_n +
	D_s + \dot{m}_s A_s + 
	\frac{\rho_P^0 V_P}{\Delta t} - S_P^\phi V_P
	\right\} \phi_P \nonumber \\
	&= 
	\Big\{ D_e - \dot{m}_e A_e \Big\} \phi_E + 
	\Big\{ D_w + \dot{m}_w A_w \Big\} \phi_W + 
	\Big\{ D_n - \dot{m}_n A_n \Big\} \phi_N + 
	\Big\{ D_s + \dot{m}_s A_s \Big\} \phi_S \nonumber \\
	&+ \left\{ S_C^\phi + \frac{\rho_P^0 \phi_P^0}{\Delta t} \right\} V_P \label{eq:discretization_equation_uds_cds_eds}
\end{align}
In order to make \eqref{eq:discretization_equation_uds_cds_eds} more tractable,
the following discretization coefficients are defined:
\begin{gather}
	\begin{align}
		a_E &= D_e - \dot{m}_e A_e \\
		a_W &= D_w + \dot{m}_w A_w \\
		a_N &= D_n - \dot{m}_n A_n \\
		a_S &= D_s + \dot{m}_s A_s
	\end{align} \\
	a_P = a_W + a_E + a_S + a_N + \frac{\rho_P^0 V_P}{\Delta t} - S_P^\phi V_P \\
	b_P = S_C^\phi V_P + \frac{\rho_P^0 \phi_P^0}{\Delta t} V_P
\end{gather}
Thereby the discretization equation is:
\begin{equation} \label{eq:small_molecule_schemes_4}
	a_P \phi_P = a_W \phi_W + a_E \phi_E + a_S \phi_S + a_N \phi_N + b_P
\end{equation}

\subsubsection{Large molecule schemes}

High--resolution schemes (HRS) such as SUDS, QUICK and SMART, not only use
adjacent nodes to the faces but also the most upstream nodes, that is to say,
involve a larger molecule. Since a larger molecule increases the memory usage
and the computational effort, it is desirable to keep it as low as possible.
Therefore, the aim is to obtain a discretization equation such as
\eqref{eq:small_molecule_schemes_4}, where only the surrounding nodes
participate, while upstream nodes are computed by different means and collected
in $b_P$. 

The first logical solution would be to use small molecule schemes, although it
must be kept in mind the lower order of the approximations. The second solution
would be to compute the upstream node value using the data of the previous
iteration and introduce this term in the equation as a contribution to $b_P$.
Nevertheless, this may lead to the divergence of the iterations since the terms
treated explicitly may be substantial \cite{ferziger2002computational5deferred}.


The solution is to compute the approximated terms with a higher order
approximation explicitly and put them on the right--hand side of equation
\eqref{eq:final_form_1}. Then a simpler approximation to these terms, for
instance one that provides a smaller molecule, is put on the left--hand side and
on the right--hand side, computing it using explicit values. Then the
right--hand side is the difference between two approximations of the same value,
hence is likely to be small. This technique is known as deferred correction, and
is used with higher--order approximations, as well as grid non--orthogonality
and correction to prevent undesirable effects in solutions
\cite{ferziger2002computational5deferred}.

Given a face $f$, the idea is approximate $\phi_f$ as
\begin{equation} \label{eq:large_molecule_schemes_1}
	\phi_f^\text{HRS} - \phi_P = 
	(\phi_f^\text{UDS} - \phi_P) + 
	(\phi_f^{\text{HRS},\ast} - \phi_f^{\text{UDS},\ast})
\end{equation}
$\phi_f^\text{HRS}$ and $\phi_f^\text{UDS}$ are the current calculated values of
$\phi$ using the chosen HRS and UDS, whereas $\phi_f^{\text{HRS},\ast}$ and
$\phi_f^{\text{UDS},\ast}$ are the computed values in the previous iteration. As
stated above, when convergence is achieved, $\phi_f^\text{HRS} =
\phi_f^{\text{HRS},\ast}$ and $\phi_f^\text{UDS} = \phi_f^{\text{UDS},\ast}$
\cite{cttc_cde_2021}. Substituting $\phi_f - \phi_P$ by $\phi_f^\text{HRS} -
\phi_P$ in \eqref{eq:general_2d_cde}
\begin{multline}
	\rho_P^0 \frac{\phi_P - \phi_P^0}{\Delta t} V_P
	+ \dot{m}_e (\phi_e^\text{HRS} - \phi_P) - \dot{m}_w (\phi_w^\text{HRS} - \phi_P) 
	+ \dot{m}_n (\phi_n^\text{HRS} - \phi_P) - \dot{m}_s (\phi_s^\text{HRS} - \phi_P) 
	= \\
	= D_e (\phi_E - \phi_P) - D_w (\phi_P - \phi_W)
	+ D_n (\phi_N - \phi_P) - D_s (\phi_P - \phi_S)
	+ (S_C^\phi + S_P^\phi \phi_P) V_P
\end{multline}
and using relation \eqref{eq:large_molecule_schemes_1}
\begin{multline}
	\rho_P^0 \frac{\phi_P - \phi_P^0}{\Delta t} V_P
	+ \dot{m}_e (\phi_e^\text{UDS} - \phi_P) - \dot{m}_w (\phi_w^\text{UDS} - \phi_P) 
	+ \dot{m}_n (\phi_n^\text{UDS} - \phi_P) - \dot{m}_s (\phi_s^\text{UDS} - \phi_P) 
	= \\
	= D_e (\phi_E - \phi_P) - D_w (\phi_P - \phi_W)
	+ D_n (\phi_N - \phi_P) - D_s (\phi_P - \phi_S)
	+ (S_C^\phi + S_P^\phi \phi_P) V_P + \\
	- \dot{m}_e (\phi_e^{\text{HRS},\ast} - \phi_e^{\text{UDS},\ast}) 
	+ \dot{m}_w (\phi_w^{\text{HRS},\ast} - \phi_w^{\text{UDS},\ast}) 
	- \dot{m}_n (\phi_n^{\text{HRS},\ast} - \phi_n^{\text{UDS},\ast}) 
	+ \dot{m}_s (\phi_s^{\text{HRS},\ast} - \phi_s^{\text{UDS},\ast})
\end{multline}
Replacing the corresponding terms with expressions \eqref{eq:uds_e} through
\eqref{eq:uds_s} and rearranging terms, the desired expression is found
\begin{equation}
	a_P \phi_P = a_W \phi_W + a_E \phi_E + a_S \phi_S + a_N \phi_N + b_P
\end{equation}
with the following coefficients:
\begin{gather}
	\begin{align}
		a_E &= D_e - \frac{\dot{m}_e - \abs{\dot{m}_e}}{2} = 
		\frac{\Gamma_e S_e}{d_{PE}} - \frac{\dot{m}_e - \abs{\dot{m}_e}}{2} \\
		a_W &= D_w + \frac{\dot{m}_w + \abs{\dot{m}_w}}{2} =
		\frac{\Gamma_w S_w}{d_{PW}} + \frac{\dot{m}_w + \abs{\dot{m}_w}}{2} \\
		a_N &= D_n - \frac{\dot{m}_n - \abs{\dot{m}_n}}{2} =
		\frac{\Gamma_n S_n}{d_{PN}} - \frac{\dot{m}_n - \abs{\dot{m}_n}}{2} \\
		a_S &= D_s + \frac{\dot{m}_s + \abs{\dot{m}_s}}{2} = 
		\frac{\Gamma_s S_s}{d_{PS}} + \frac{\dot{m}_s + \abs{\dot{m}_s}}{2}
	\end{align} \\
	a_P = a_W + a_E + a_S + a_N + \frac{\rho_P^0 V_P}{\Delta t} - S_P^\phi V_P \\
	\begin{align}
		b_P = \frac{\rho_P^0 \phi_P^0}{\Delta t} V_P + S_C^\phi V_P 
		&- \dot{m}_e (\phi_e^{\text{HRS},\ast} - \phi_e^{\text{UDS},\ast}) 
		+ \dot{m}_w (\phi_w^{\text{HRS},\ast} - \phi_w^{\text{UDS},\ast}) \nonumber \\
		&- \dot{m}_n (\phi_n^{\text{HRS},\ast} - \phi_n^{\text{UDS},\ast}) 
		+ \dot{m}_s (\phi_s^{\text{HRS},\ast} - \phi_s^{\text{UDS},\ast})
	\end{align}
\end{gather}





