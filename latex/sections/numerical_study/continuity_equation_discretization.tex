
\subsection{Discretization of the continuity equation}

As we have seen, the continuity equation in differential form is
\begin{equation}
	\pdv{\rho}{t} + \div(\rho \vb{v}) = 0 \quad (\vb{x},t) \in \Omega \times I
\end{equation}
Since the above relation is true on $\Omega \times I$, fixing one time $t \in I$ and integrating over a control volume $\cv{P} \subset \Omega$ gives
\begin{equation} \label{eq:discretization_continuity_equation_1}
	\int_{\cv{P}} \pdv{\rho}{t} \dd{\vb{x}} + \int_{\cv{P}} \div(\rho \vb{v}) \dd{\vb{x}} = 0
\end{equation}
Let $\cs{P} = \partial \cv{P}$ be the control surface, \ie the boundary of the control volume. Then applying the divergence theorem on the second term of equation \eqref{eq:discretization_continuity_equation_1},
\begin{equation} \label{eq:discretization_continuity_equation_2}
	\int_{\cv{P}} \pdv{\rho}{t} \dd{\vb{x}} + 
	\int_{\cs{P}} \rho \vb{v} \vdot \vb{n} \dd{S} = 0
\end{equation}
With the aim of simplifying the first term of \eqref{eq:discretization_continuity_equation_2}, the average density of the control volume is defined in the following way:
\begin{equation}
	\overline{\rho}_P = \frac{1}{V_P} \int_{\cv{P}} \rho \dd{\vb{x}}
\end{equation}
Introducing this relation in equation \eqref{eq:discretization_continuity_equation_2},
\begin{equation} \label{eq:discretization_continuity_equation_3}
	\frac{\dd \overline{\rho}_P}{\dd{t}} V_P + 
	\int_{\cs{P}} \rho \vb{v} \vdot \vb{n} \dd{S} = 0
\end{equation}
The mass flow term can be further simplified if a cartesian mesh is being used. In case of a 2D--mesh, the control surface can be partitioned into four different faces, namely, the east, west, north and south faces. Hence the control surface is $\cs{P} = \cs{Pe} \cup \cs{Pw} \cup \cs{Pn} \cup \cs{Ps}$ and the mass flow term can be expressed as
\begin{align}
	\int_{\cs{P}} \rho \vb{v} \vdot \vb{n} \dd{S} 
	&= 
	\underbrace{\int_{\cs{Pe}} \rho \vb{v} \vdot \vb{n} \dd{S}}_{\dot{m}_e}
	+ \underbrace{\int_{\cs{Pw}} \rho \vb{v} \vdot \vb{n} \dd{S}}_{-\dot{m}_w}
	+ \underbrace{\int_{\cs{Pn}} \rho \vb{v} \vdot \vb{n} \dd{S}}_{\dot{m}_n}
	+ \underbrace{\int_{\cs{Ps}} \rho \vb{v} \vdot \vb{n} \dd{S}}_{-\dot{m}_s} \nonumber \\
	&= \dot{m}_e - \dot{m}_w + \dot{m}_n - \dot{m}_s \label{eq:discretization_continuity_equation_7}
\end{align}
Since evaluating each integral may be computationally expensive or impossible, the following approach is followed. Given a face $f$, the normal outer vector is constant on $\cs{Pf}$. Indeed, if $\vb{n}_f$ denotes the normal outer vector to face $f$, then $\vb{n}_e = \vb{i}$, $\vb{n}_w = -\vb{i}$, $\vb{n}_n = \vb{j}$ and $\vb{n}_s = -\vb{j}$. Since $\vb{v} = u \vb{i} + v \vb{j}$, the dot products are $\vb{v} \vdot \vb{n}_e = u$, $\vb{v} \vdot \vb{n}_w = -u$ and so on. Moreover, the integrand $(\rho \vb{v} \vdot \vb{n})_f$ can be approximated by the value each term takes at the face center, \ie
\begin{equation}
	(\rho \vb{v} \vdot \vb{n})_f \approx \rho_f (\vb{v} \vdot \vb{n})_f
\end{equation}
Therefore the integral over $\cs{Pe}$ on equation \eqref{eq:discretization_continuity_equation_7} may be simplified as follows:
\begin{align}
	\int_{\cs{Pe}} \rho \vb{v} \vdot \vb{n} \dd{S} \approx
	\int_{\cs{Pe}} (\rho \vb{v} \vdot \vb{n})_e \dd{S} \approx
	\int_{\cs{Pe}} \rho_e (\vb{v} \vdot \vb{n})_e) \dd{S} = 
	\int_{\cs{Pe}} \rho_e u_e \dd{S} = 
	\rho_e u_e S_{Pe} \eqqcolon \dot{m}_e
\end{align}
The same simplifications are applied to compute the remaining integral. Defining $\dot{m}_w$ and $\dot{m}_s$ as the negative integral makes the mass flow terms be positive in the positive coordinate direction. Introducing these in \eqref{eq:discretization_continuity_equation_3} yields
\begin{equation} \label{eq:discretization_continuity_equation_8}
	\frac{\dd \overline{\rho}_P}{\dd{t}} V_P 
	+ \dot{m}_e - \dot{m}_w + \dot{m}_n - \dot{m}_s = 0
\end{equation}
The average density of the control volume is roughly the density at the discretization node, that is, $\overline{\rho}_P \approx \rho_P$. Integrating \eqref{eq:discretization_continuity_equation_8} over the time interval $[t^n, t^{n+1}]$ gives
\begin{equation} \label{eq:discretization_continuity_equation_5}
	V_P \int_{t^n}^{t^{n+1}} \frac{\dd \rho_P}{\dd{t}} \dd{t} + 
	\int_{t^n}^{t^{n+1}} ( \dot{m}_e - \dot{m}_w + \dot{m}_n - \dot{m}_s ) \dd{t} = 0
\end{equation}
The first term of \eqref{eq:discretization_continuity_equation_5} has a straightforward simplification applying a corollary of the fundamental theorem of calculus. Regarding the second term, numerical integration is used,
\begin{align} 
	(\rho_P^{n+1} - \rho_P^n) V_P
	&+ \beta (\dot{m}_e^{n+1} - \dot{m}_w^{n+1} + \dot{m}_n^{n+1} - \dot{m}_s^{n+1}) (t^{n+1} - t^n) 
	\nonumber \\
	&+ (1 - \beta) (\dot{m}_e^n - \dot{m}_w^n + \dot{m}_n^n - \dot{m}_s^n) (t^{n+1} - t^n) = 0
	\label{eq:discretization_continuity_equation_6}
\end{align}
where $\beta \in \{ 0, \frac{1}{2}, 1 \}$ depends on the chosen integration scheme. For the sake of simplicity, superindex $n+1$ shall be dropped and the time instant $n$ will be denoted by the superindex $0$. Assuming a uniform time step $\Delta t$, the resulting discretized continuity equation is
\begin{equation}
	\frac{\rho_P - \rho_P^0}{\Delta t} V_P 
	+ \beta (\dot{m}_e - \dot{m}_w + \dot{m}_n - \dot{m}_s)
	+ (1 - \beta) (\dot{m}_e^0 - \dot{m}_w^0 + \dot{m}_n^0 - \dot{m}_s^0) = 0
\end{equation}
Finally, when an implicit scheme is selected for the time integration,
\begin{equation} \label{eq:continuity_equation_2d_discretized}
	\frac{\rho_P - \rho_P^0}{\Delta t} V_P + 
	\dot{m}_e - \dot{m}_w + 
	\dot{m}_n - \dot{m}_s = 0
\end{equation}

If a 3D--mesh is being used, the contributions of top and bottom faces must be considered. The control surface is the union $\cs{P} = \cs{Pe} \cup \cs{Pw} \cup \cs{Pn} \cup \cs{Ps} \cup \cs{Pt} \cup \cs{Pb}$, hence equation \eqref{eq:continuity_equation_2d_discretized} incorporates two new terms
\begin{equation} \label{eq:continuity_equation_3d_discretized}
	\frac{\rho_P - \rho_P^0}{\Delta t} V_P + 
	\dot{m}_e - \dot{m}_w + \dot{m}_n - \dot{m}_s + \dot{m}_t - \dot{m}_b = 0
\end{equation}










