
\subsection{Discretization of the continuity equation}

As we have seen before, the continuity equation in differential form is
\begin{equation}
	\pdv{\rho}{t} + \div(\rho \vb{v}) = 0 \quad (\vb{x},t) \in \Omega \times I
\end{equation}
Since the above relation is true on $\Omega \times I$, fixing one time $t \in I$ and integrating over a control volume $\mathcal{V} \subset \Omega$ yields
\begin{equation} \label{eq:discretization_continuity_equation_1}
	\int_{\mathcal{V}} \pdv{\rho}{t} \dd{\vb{x}} + \int_{\mathcal{V}} \div(\rho \vb{v}) \dd{\vb{x}} = 0
\end{equation}
Let $\mathcal{S} = \partial \mathcal{V}$ be the control surface, \ie the boundary of the control volume. Then applying the divergence theorem on the second term of equation \eqref{eq:discretization_continuity_equation_1} gives
\begin{equation} \label{eq:discretization_continuity_equation_2}
	\int_{\mathcal{V}} \pdv{\rho}{t} \dd{\vb{x}} + 
	\int_{\mathcal{S}} \rho \vb{v} \vdot \vb{n} \dd{S} = 0
\end{equation}
With the aim of simplifying the first term of \eqref{eq:discretization_continuity_equation_2}, we define the average density of the control volume as
\begin{equation}
	\overline{\rho} = \frac{1}{V} \int_{\mathcal{V}} \rho \dd{\vb{x}}
\end{equation}
Introducing this relation in equation \eqref{eq:discretization_continuity_equation_2} gives
\begin{equation} \label{eq:discretization_continuity_equation_3}
	\frac{\dd \overline{\rho}}{\dd{t}} V + 
	\int_{\mathcal{S}} \rho \vb{v} \vdot \vb{n} \dd{S} = 0
\end{equation}
The mass flow term can be further simplified if we are using a cartesian mesh. In case of a 2D--mesh, the control surface can be partitioned into four different faces, namely, the east, west, north and south faces. In this context the control surface is $\mathcal{S} = \mathcal{S}_e \cup \mathcal{S}_w \cup \mathcal{S}_n \cup \mathcal{S}_s$ and we may express the mass flow term as
\begin{equation}
	\int_{\mathcal{S}} \rho \vb{v} \vdot \vb{n} \dd{S} = 
	\sum_{i} \int_{\mathcal{S}_i} \rho \vb{v} \vdot \vb{n} \dd{S} = 
	\dot{m}_e + \dot{m}_w + \dot{m}_n + \dot{m}_s
\end{equation}
If we use a 3D--mesh, we must consider the contributions of top and bottom faces. The control surface is the union $\mathcal{S} = \mathcal{S}_e \cup \mathcal{S}_w \cup \mathcal{S}_n \cup \mathcal{S}_s \cup \mathcal{S}_t \cup \mathcal{S}_b$, and therefore the mass flow incorporates two new terms
\begin{equation}
	\int_{\mathcal{S}} \rho \vb{v} \vdot \vb{n} \dd{S} = 
	\sum_{i} \int_{\mathcal{S}_i} \rho \vb{v} \vdot \vb{n} \dd{S} = 
	\dot{m}_e + \dot{m}_w + \dot{m}_n + \dot{m}_s + \dot{m}_n + \dot{m}_b
\end{equation}
In both cases equation \eqref{eq:discretization_continuity_equation_3} is rewritten in the following way
\begin{equation} \label{eq:discretization_continuity_equation_4}
	\frac{\dd \overline{\rho}}{\dd{t}} V + \sum_i \dot{m}_i = 0
\end{equation}
The average density of the control volume is roughly the density at the discretization node, that is, $\overline{\rho} \approx \rho$. Integrating \eqref{eq:discretization_continuity_equation_4} over the time interval $[t^n, t^{n+1}]$ gives
\begin{equation} \label{eq:discretization_continuity_equation_5}
	V \int_{t^n}^{t^{n+1}} \frac{\dd \overline{\rho}}{\dd{t}} \dd{t} + 
	\int_{t^n}^{t^{n+1}} \sum_i \dot{m}_i \dd{t} = 0
\end{equation}
The first term of \eqref{eq:discretization_continuity_equation_5} has a straightforward simplification applying a corollary of the fundamental theorem of calculus. Regarding the second term, we use numerical integration which, in general, gives a non--exact result,
\begin{equation} \label{eq:discretization_continuity_equation_6}
	V (\rho^{n+1} - \rho^n) + 
	\left( \beta \sum_i \dot{m}_i^{n+1} + (1 - \beta) \sum_i \dot{m}_i^{n} \right) (t^{n+1} - t^n) = 0
\end{equation}
where $\beta \in \{ 0, \frac{1}{2}, 1 \}$ depends on the chosen integration scheme. For the sake of simplicity, we shall drop the superindex $n+1$ and the time instant $n$ will be denoted by the superindex $0$. Assuming a uniform time step $\Delta t$, the resulting discretized continuity equation is
\begin{equation}
	\frac{\rho - \rho^0}{\Delta t} V + \beta \sum_i \dot{m}_i + (1 - \beta) \sum_i \dot{m}_i^0 = 0
\end{equation}





