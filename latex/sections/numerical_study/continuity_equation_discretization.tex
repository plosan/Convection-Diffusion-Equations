
\subsection{Discretization of the continuity equation}

As seen before, the continuity equation in differential form is
\begin{equation}
	\pdv{\rho}{t} + \div(\rho \vb{v}) = 0 \quad (\vb{x},t) \in \Omega \times I
\end{equation}
Since the above relation is true on $\Omega \times I$, fixing one time $t \in I$ and integrating over a control volume $\cv{P} \subset \Omega$ yields
\begin{equation} \label{eq:discretization_continuity_equation_1}
	\int_{\cv{P}} \pdv{\rho}{t} \dd{\vb{x}} + \int_{\cv{P}} \div(\rho \vb{v}) \dd{\vb{x}} = 0
\end{equation}
Let $\cs{P} = \partial \cv{P}$ be the control surface, \ie the boundary of the control volume. Then applying the divergence theorem on the second term of equation \eqref{eq:discretization_continuity_equation_1}
\begin{equation} \label{eq:discretization_continuity_equation_2}
	\int_{\cv{P}} \pdv{\rho}{t} \dd{\vb{x}} + 
	\int_{\cs{P}} \rho \vb{v} \vdot \vb{n} \dd{S} = 0
\end{equation}
With the aim of simplifying the first term of \eqref{eq:discretization_continuity_equation_2}, the average density of the control volume is defined as
\begin{equation}
	\overline{\rho}_P = \frac{1}{V_P} \int_{\cv{P}} \rho \dd{\vb{x}}
\end{equation}
Introducing this relation in equation \eqref{eq:discretization_continuity_equation_2} gives
\begin{equation} \label{eq:discretization_continuity_equation_3}
	\frac{\dd \overline{\rho}_P}{\dd{t}} V_P + 
	\int_{\cs{P}} \rho \vb{v} \vdot \vb{n} \dd{S} = 0
\end{equation}
The mass flow term can be further simplified if a cartesian mesh is being used. In case of a 2D--mesh, the control surface can be partitioned into four different faces, namely, the east, west, north and south faces. In this context the control surface is $\cs{P} = \cs{Pe} \cup \cs{Pw} \cup \cs{Pn} \cup \cs{Ps}$ and the mass flow term may be expressed as
\begin{equation}
	\int_{\cs{P}} \rho \vb{v} \vdot \vb{n} \dd{S} = 
	\sum_{i} \int_{\cs{Pi}} \rho \vb{v} \vdot \vb{n} \dd{S} = 
	\dot{m}_e + \dot{m}_w + \dot{m}_n + \dot{m}_s
\end{equation}
If on the other hand a 3D--mesh is being used, the contributions of top and bottom faces must be considered. The control surface is the union $\cs{P} = \cs{Pe} \cup \cs{Pw} \cup \cs{Pn} \cup \cs{Ps} \cup \cs{Pt} \cup \cs{Pb}$, and therefore the mass flow incorporates two new terms
\begin{equation}
	\int_{\cs{P}} \rho \vb{v} \vdot \vb{n} \dd{S} = 
	\sum_{i} \int_{\cs{Pi}} \rho \vb{v} \vdot \vb{n} \dd{S} = 
	\dot{m}_e + \dot{m}_w + \dot{m}_n + \dot{m}_s + \dot{m}_n + \dot{m}_b
\end{equation}
This allows writing equation \eqref{eq:discretization_continuity_equation_3} in the following way
\begin{equation} \label{eq:discretization_continuity_equation_4}
	\frac{\dd \overline{\rho}_P}{\dd{t}} V_P + \sum_i \dot{m}_i = 0
\end{equation}
The average density of the control volume is roughly the density at the discretization node, that is, $\overline{\rho}_P \approx \rho_P$. Integrating \eqref{eq:discretization_continuity_equation_4} over the time interval $[t^n, t^{n+1}]$ gives
\begin{equation} \label{eq:discretization_continuity_equation_5}
	V_P \int_{t^n}^{t^{n+1}} \frac{\dd \overline{\rho}_P}{\dd{t}} \dd{t} + 
	\int_{t^n}^{t^{n+1}} \sum_i \dot{m}_i \dd{t} = 0
\end{equation}
The first term of \eqref{eq:discretization_continuity_equation_5} has a straightforward simplification applying a corollary of the fundamental theorem of calculus. Regarding the second term, numerical integration is applied,
\begin{equation} \label{eq:discretization_continuity_equation_6}
	(\rho_P^{n+1} - \rho_P^n) V_P + 
	\left( \beta \sum_i \dot{m}_i^{n+1} + (1 - \beta) \sum_i \dot{m}_i^{n} \right) (t^{n+1} - t^n) = 0
\end{equation}
where $\beta \in \{ 0, \frac{1}{2}, 1 \}$ depends on the chosen integration scheme. For the sake of simplicity, superindex $n+1$ shall be dropped and the time instant $n$ will be denoted by the superindex $0$. Assuming a uniform time step $\Delta t$, the resulting discretized continuity equation is
\begin{equation}
	\frac{\rho_P - \rho_P^0}{\Delta t} V_P + \beta \sum_i \dot{m}_i + (1 - \beta) \sum_i \dot{m}_i^0 = 0
\end{equation}





