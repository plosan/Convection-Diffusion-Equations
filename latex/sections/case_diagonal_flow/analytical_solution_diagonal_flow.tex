
\subsection{Analytical solution}

\subsubsection{Analytical solution for \texorpdfstring{$\mathrm{Pe} = \infty$}{infinite Péclet's number}}

Péclet's number is defined as the ratio between \colorbox{red}{transport and heat difussion}, $\mathrm{Pe} = \rho u L / \Gamma$. Whenever $\mathrm{Pe} \to +\infty$, it implies $\Gamma \to 0$ since infinite values for the density, velocity or characteristic length make no physical sense. Therefore the difussion coefficient tends to $0$, which means the Laplacian term is negligible. Under this physical intuition, dividing the PDE from \eqref{eq:diagonal_case_cauchy_problem} results in the following Cauchy problem:
\begin{equation} \label{eq:diagonal_case_cauchy_problem_infinite_peclet}
	\left\{
	\begin{aligned}
		&\pdv{\phi}{x} + \pdv{\phi}{y} = 0 &
		&\text{in } \Omega = (0,L) \times (0,L) \\
		&\phi = \phi_\text{low} &
		&\text{on } [0,L) \times \{ 0 \} \cup \{ L \} \times [0,L) \\
		&\phi = \phi_\text{high} &
		&\text{on } \{ 0 \} \times (0,L] \cup (0,L] \times \{ L \}
	\end{aligned}
	\right.
\end{equation}
The PDE from \eqref{eq:diagonal_case_cauchy_problem_infinite_peclet} is known as the transport equation, which is a first order linear PDE. In our case it has constant coefficients, making it easier to solve analitically. So as to find the analytical solution to \eqref{eq:diagonal_case_cauchy_problem_infinite_peclet}, we will follow a geometric approach as it is more intuitive. 

by dividing the PDE from \eqref{eq:diagonal_case_cauchy_problem} by 

\subsubsection{Analytical solution for \texorpdfstring{$\mathrm{Pe} = 0$}{zero Péclet's number}}

\subsubsection{General problem}