\subsubsection{Analytical solution for \texorpdfstring{$\peclet = 0$}{zero Péclet's number}}

Now we consider the problem \eqref{eq:diagonal_case_cauchy_problem} when $\peclet \to 0$. Since $\rho > 0$ and $L > 0$, the fact that Péclet's number is close to zero implies that velocity $u$ is close to zero. In the extreme case when $u = 0$, there is no transport, therefore $\peclet = 0$ and problem \eqref{eq:diagonal_case_cauchy_problem} becomes
\begin{equation} \label{eq:diagonal_case_cauchy_problem_zero_peclet}
	\left\{
	\begin{aligned}
		&\Delta \phi = 0 &
		&\text{in } \Omega \\
		&\phi = g &
		&\text{on } \partial \Omega
	\end{aligned}
	\right.
\end{equation}
which is Laplace's problem in the square $\Omega$. 