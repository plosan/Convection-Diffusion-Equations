\subsubsection{Weak analytical solution for \texorpdfstring{$\mathrm{Pe} =
\infty$}{infinite Péclet's number}}


As we have seen in the previous subsection, problem 

\begin{definition}
	A function $\psi \colon \overline{\Omega} \rightarrow \real$ is said to be a
	weak solution of problem
	\eqref{eq:diagonal_case_cauchy_problem_infinite_peclet} if for all test
	functions $\psi \in \mathcal{C}^1_c (\overline{\Omega})$ the following
	integral equation is satisfied:
	\begin{equation}
		\int_\Omega 
	\end{equation}
	
	A function $\psi \colon \overline{\Omega} \rightarrow \real$
	is said a weak solution of
	\eqref{eq:diagonal_case_cauchy_problem_infinite_peclet} if 
	\[
		\int_\Omega 
	\]
\end{definition}


Notice that each characteristic starting on $C_1$ ends on $C_1$, and the same
holds for $C_2$. 

y definition of the Cauchy problem, $\phi$ is
constant on $C_1$ and on $C_2$. Therefore the value of $\phi$ on the
characteristic $x - y = c$ is the value that $g$ takes on the part of the
boundary the characteristic intersects.