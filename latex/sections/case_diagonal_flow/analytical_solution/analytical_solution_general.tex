\subsubsection{General problem}

Hereinafter we consider problem \eqref{eq:diagonal_case_cauchy_problem} with $0
< \mathrm{Pe} < +\infty$ with $\phi_\text{low} < \phi_\text{high}$. To begin, we
focus on the existence of classical solution.

\begin{definition}
	A classical solution to problem
	\eqref{eq:diagonal_case_cauchy_problem_infinite_peclet} is a function $\phi
	\colon \overline{\Omega} \rightarrow \real$ that satisfies:
	\begin{enumerate}[label={(\roman*)}, topsep=0pt]
		\item $\phi \in \mathcal{C}^2(\Omega) \cap
		\mathcal{C}(\overline{\Omega})$,
		\item $\phi$ satisfies the PDE, and
		\item $\phi$ satisfies the boundary conditions.
	\end{enumerate}
\end{definition}

\noindent
The function $g$ giving the boundary conditions is not continuous at $(0,0)$ nor
at $(L, L)$ unless $\phi_\text{low} = \phi_\text{high}$. Therefore problem
\eqref{eq:diagonal_case_cauchy_problem} cannot have a classical solution.
Nonetheless it might have a solution in the weak sense.

Before studying the theorem that deals with the existence of a weak 

\begin{definition}
	contenidos...
\end{definition}