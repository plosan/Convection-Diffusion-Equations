
\subsubsection{Weak solution for \texorpdfstring{$0 \leq \peclet <
\infty$}{finite Péclet's number}}

The theory that studies the existence and uniqueness of weak solutions to Cauchy
problems involving second--order elliptic equations requires a prior knowledge
in measure theory and functional analysis. The interested reader can consult
appendix \ref{ap:measure_theory} for a quick reference in some basic concepts of
measure theory. We will not introduce functional analysis concepts in order not
to complicate the exposition needlessly. Rather than focusing on problem
\eqref{eq:diagonal_case_cauchy_problem}, we shall study the theory for a general
second--order elliptic equation and then particularize to our case. The
reference for this subsection is Lawrence C. Evans' excellent book on Partial
Differential Equations \cite{evans1998pde}, in particular chapter 6.

\subsubsection*{General theory for weak solutions}

Consider the Cauchy problem
\begin{equation} \label{eq:second_order_elliptic_problem}
	\left\{
		\begin{aligned}
			L u &= f & &\text{in } U \\
			u &= 0 & &\text{on } \partial U
		\end{aligned}
	\right.
\end{equation}
where $U \subset \real^n$ is an open bounded subset, $u \colon \overline{U}
\rightarrow \real$ is the unknown function (shall not be confused with the
$x$--component of the velocity field) and $f \colon U \rightarrow \real$ is a
given function. $L$ is a second--order partial differential operator having one
of the two following forms
\begin{gather}
	L u = - \sum_{i,j=1}^n (a^{ij}(x) u_{x_i})_{x_j} + \sum_{i=1}^n b^i(x) u_{x_i} + c(x) u \label{eq:divergence_form}\\
	L u = - \sum_{i,j=1}^n a^{ij}(x) u_{x_i x_j} + \sum_{i=1}^n b^i(x) u_{x_i} + c(x) u	
	\label{eq:nondivergence_form}
\end{gather}
being $a^{ij}, \, b^i, \, c$ for $i, j = 1, \ldots, n$ given functions.
Aditionally, we shall assume $a^{ij}, \, b^i, \, c \in L^\infty(U)$ for all $i,
j = 1, \ldots, n$ and $f \in L^2(U)$. We say $L$ is in divergence form when is
given by \eqref{eq:divergence_form}, whereas $L$ is in non--divergence form when
expressed by \eqref{eq:nondivergence_form}.

\begin{prop}
	Whenever $a^{ij} \in \mathcal{C}^1(U)$ for $i, j = 1, \ldots, n$, a partial
	differential operator in divergence form can be rewritten in non--divergence
	form and viceversa.
\end{prop}
\begin{proof}
	Assume $L$ is given in divergence form, then
	\begin{align*}
		L u &= -\sum_{i,j=1}^n (a^{ij}(x) u_{x_i})x_j + \sum_{i=1}^n b^i(x) u_{x_i} + c(x) u \\
		&=
		-\sum_{i,j=1}^n \left\{ a^{ij}(x) u_{x_i x_j} + a_{x_j}^{ij}(x) u_{x_i} \right\} + \sum_{i=1}^n b^i(x) u_{x_i} + c(x) u \\
		&= 
		-\sum_{i,j=1}^na^{ij}(x) u_{x_i x_j} + \sum_{i=1}^n \left\{ b^i(x) - \sum_{j=1}^n a_{x_j}^{ij}(x) \right\} u_{x_i} + c(x) u
	\end{align*}
	which is in non--divergence form. Conversely, assume $L$ is given in non--divergence form, thus
	\begin{align*}		
		L u &= - \sum_{i,j=1}^n a^{ij}(x) u_{x_i x_j} + \sum_{i=1}^n b^i(x) u_{x_i} + c(x) u \\
		&=
		- \sum_{i,j=1}^n \left\{ a^{ij}(x) u_{x_i x_j} + a_{x_j}^{ij}(x) u_{x_i} - a_{x_j}^{ij}(x) u_{x_i} \right\}
		+ \sum_{i=1}^n b^i(x) u_{x_i} + c(x) u \\
		&=
		- \sum_{i,j=1}^n (a^{ij}(x) u_{x_i})_{x_j} + \sum_{i=1}^n \left\{ b^i(x) + \sum_{j=1}^n a_{x_j}^{ij}(x) \right\} u_{x_i} + c(x) u \\
	\end{align*}
	giving the divergence form of $L$.
\end{proof}

\begin{definition}
	\begin{enumerate}[label={(\roman*)}, topsep=0pt]
		\item[]
		\item We say the partial differential operator $L$ is symmetric provided 
		\begin{equation} \label{eq:operator_L_symmetry}
			a^{ij} = a^{ji}
		\end{equation}
		for all $i, j = 1, \ldots, n$.
		\item We say the partial differential operator $L$ is (uniformly)
		elliptic if there exists a constant $\theta > 0$ such that
		\begin{equation} \label{eq:operator_L_uniformly_elliptic}
			\sum_{i,j=1}^n a^{ij}(x) \, \xi_i \xi_j \geq \theta \norm{\xi}^2
		\end{equation}
	\end{enumerate}
	for a.e. $x \in U$ and all $\xi \in \real^n$.
\end{definition}

Hereinafter we will assume operator $L$ satisfies both the symmetry and the
uniform ellipticity conditions. Before giving the definition of weak solution,
we shall gain some intuition about it. Suppose $L$ is given in divergence form.
Let us assume that $u$ is a smooth solution to
\eqref{eq:second_order_elliptic_problem}, \ie $u$ is a differentiable enough
function. Let $v \in \mathcal{C}_c^\infty(U)$ be a smooth test function, where
$\mathcal{C}_c^\infty(U)$ is the set of infinitely differentiable functions with
compact support in $U$. We multiply $L u = f$ by $v$ and integrate over $U$:
\begin{equation} \label{eq:second_order_elliptic_by_parts_1}
	\int_U - \sum_{i,j=1}^n (a^{ij}(x) u_{x_i})_{x_j} v \dd{x} + 
	\int_U \sum_{i=1}^n b^i(x) u_{x_i} v \dd{x} + 
	\int_U c(x) u v \dd{x} = 
	\int_U f v \dd{x}
\end{equation}
We may rewrite the first term on the left--hand side integrating by parts
\begin{align}
	\int_U - \sum_{i,j=1}^n (a^{ij}(x) u_{x_i})_{x_j} v \dd{x} &= 
	\int_U \sum_{i,j=1}^n a^{ij}(x) u_{x_i} v_{x_j} \dd{x} - 
	\int_{\partial U} \sum_{i,j=1}^n a^{ij}(x) u_{x_i} v \nu^j \dd{\sigma} \nonumber \\
	&= \int_U \sum_{i,j=1}^n a^{ij}(x) u_{x_i} v_{x_j} \dd{x} \label{eq:second_order_elliptic_by_parts_2}
\end{align}
In \eqref{eq:second_order_elliptic_by_parts_2}, $\nu^j$ denotes the $j$--th
component of the exterior normal vector to $\partial U$ and $\dd{\sigma}$ is the
``surface'' element of $\partial U$. The integral over $\partial U$ is zero as
$v$ vanishes on it as a consequence of being in $\mathcal{C}^\infty_c(U)$.
Introducing \eqref{eq:second_order_elliptic_by_parts_2} in
\eqref{eq:second_order_elliptic_by_parts_1} yields
\begin{equation} \label{eq:second_order_elliptic_by_parts_3}
	\int_U \sum_{i,j=1}^n a^{ij} u_{x_i} v_{x_j} + \sum_{i=1}^n b^i u_{x_i} v + c u v \dd{x} =
	\int_U f v \dd{x}
\end{equation}
The left--hand side of \eqref{eq:second_order_elliptic_by_parts_3} is the
bilinear form
\begin{equation}
	B[u,v] = 
	\int_U \sum_{i,j=1}^n a^{ij} u_{x_i} v_{x_j} + \sum_{i=1}^n b^i u_{x_i} v + c u v \dd{x}
\end{equation}
for $u, v \in H^1_0(U)$, and is associated to the operator $L$ in divergence
form. Thus equality \eqref{eq:second_order_elliptic_by_parts_3} is rewritten as
\begin{equation}
	B[u,v] = (f, v)
\end{equation}
where $(f, v) = \int_U f v \dd{x}$ is the scalar product in $L^2(U)$. Once we
have introduced $B[\cdot,\cdot]$, we can define the concept of weak solution.

\begin{definition} \label{def:weak_solution}
	A function $u \in H^1_0(U)$ is weak solution of the boundary value problem
	\eqref{eq:second_order_elliptic_problem} if
	\begin{equation} \label{eq:definition_weak_solution}
		B[u,v] = (f,v) \quad \text{for all } v \in H^1_0(U)
	\end{equation}
\end{definition}

\noindent
The intuition behind the definition of weak solution is the following: when a
function $u \in H^1_0(U)$ is fixed, and equality
\eqref{eq:definition_weak_solution} holds for any other function $v \in
H^1_0(U)$, following the development we have made, necessarily $u$ is a solution
to $L u = f$ in $U \subset \real^n$. The theorem that makes this intuition
precise, although some further hypothesis on $B$ are necessary, is the
Lax--Milgram theorem. 

Before stating the theorem on the existence and uniqueness of solution to
\eqref{eq:second_order_elliptic_problem}, other results such as Lax--Milgram's
theorem or energy estimates are necessary, however we shall not present them
here so as not to complicate the exposition. The interested reader can refer to
\cite{evans1998pde}. The following is Theorem 4 from section 6.2 of Evan's book.

\begin{theorem}[Second Existence Theorem for weak solutions] \label{teo:weak_solution}	
	\begin{enumerate}[label={(\arabic*)}, topsep=0pt]
		\item[]
		\item Precisely one of the following statements holds:
		\begin{enumerate}[label={(1.\arabic*)}, topsep=0pt]
			\item For each $f \in L^2(U)$ there exists a unique weak solution $u$ of
			the boundary value problem \label{item:alpha}
			\begin{equation} \label{eq:boundary_value_problem_alpha}
				\left\{
					\begin{aligned}
						L u &= f & &\text{in } U \\
						u &= 0 & &\text{on } \partial U
					\end{aligned}
				\right.
			\end{equation}
			\item There exists a weak solution $u \not\equiv 0$ of the homogeneous
			problem \label{item:beta}
			\begin{equation} \label{eq:boundary_value_problem_beta}
				\left\{
					\begin{aligned}
						L u &= 0 & &\text{in } U \\
						u &= 0 & &\text{on } \partial U
					\end{aligned}
				\right.
			\end{equation}
		\end{enumerate}
		\item Furthermore, should assertion \ref{item:beta} hold, the dimension
		of the subspace $N \subset H_0^1(U)$ of weak solutions of
		\eqref{eq:boundary_value_problem_beta} is finite and equals the
		dimension of the subspace $N^\ast \subset H_0^1(U)$ of weak solutions of
		\begin{equation}			
			\left\{
				\begin{aligned}
					L^\ast v &= 0 & &\text{in } U \\
					v &= 0 & &\text{on } \partial U
				\end{aligned}
			\right.
		\end{equation}
		\item Finally, the boundary--value problem
		\eqref{eq:boundary_value_problem_alpha} has a weak solution if and only if 
		\begin{equation}
			(f, v) = 0 \quad \text{for all } v \in N^\ast
		\end{equation}
	\end{enumerate}
\end{theorem}


\begin{theorem}[Infinite differentiability in the interior]
	Assume
	\[
		a^{ij}, \, b^i, \, c \in \mathcal{C}^\infty(U) \quad 
		(i,j = 1, \ldots, n)	
	\]
	and
	\[
		f \in \mathcal{C}^\infty(U)
	\]
	Suppose $u \in H^1(U)$ is a weak solution of the elliptic PDE
	\[
		\begin{aligned}
			L u &= f & &\text{in } U
		\end{aligned}	
	\]
	Then
	\[
		u \in \mathcal{C}^\infty(U)
	\]
\end{theorem}

\subsubsection*{Weak solution for \texorpdfstring{$0 \leq \peclet <
\infty$}{finite Péclet's number}}

Now we shall apply the previous theory to problem
\eqref{eq:diagonal_case_cauchy_problem}. Observe that the PDE is equivalent to
\begin{equation} \label{eq:weak_solution_pde_1}
	-\Delta \phi + (\phi_x + \phi_y) \beta \, \peclet = 
	-\phi_{xx} - \phi_{yy} + (\phi_x + \phi_y) \beta \, \peclet = 0
\end{equation}
So as to express \eqref{eq:weak_solution_pde_1} with an operator $L$ such as the
ones described above, we define
\begin{equation} \label{eq:weak_solution_pde_2}
	\begin{aligned}
		a^{11} &= 1 	& 	a^{12} &= 0 	& 	b^1 &= \beta \, \peclet	& 	c &= 0 \\
		a^{21} &= 0 	& 	a^{22} &= 1 	& 	b^2 &= \beta \, \peclet
	\end{aligned}
\end{equation}
therefore the operator $L$ encoding \eqref{eq:weak_solution_pde_1} is simply
\begin{equation} \label{eq:weak_solution_pde_3}
	L \phi = 
	-\sum_{i,j=1}^2 (a^{ij} \phi_{x_i})_{x_j} + \sum_{i=1}^n b^i \phi_{x_i} + c \phi =
	-\phi_{xx} - \phi_{yy} + (\phi_x + \phi_y) \beta \, \peclet 
\end{equation}
where $x_1 \equiv x, \, x_2 \equiv y$. Moreover, since the functions in
\eqref{eq:weak_solution_pde_2} are constant, these belong to $L^\infty(\Omega)$
and the divergence form and non--divergence form of
\eqref{eq:weak_solution_pde_3} are the same. It is obvious that $L$ is
symmetric, as $a^{12} = a^{21} = 0$, and it is uniformly elliptic with $\theta = 1$ as a result of
\begin{equation}
	\sum_{i,j=1}^2 a^{ij} \xi_i \xi_j = 
	{\xi_1}^2 + {\xi_2}^2 = \theta \norm{\xi}^2
\end{equation}
for all $\xi \in \real^2$.

Now we shall transform the boundary--value problem
\eqref{eq:diagonal_case_cauchy_problem} into a problem having the form
\eqref{eq:second_order_elliptic_problem}. Observe that the boundary condition in
\eqref{eq:second_order_elliptic_problem} is homogeneous, \ie $u = 0$ on
$\partial U$, whilst in \eqref{eq:diagonal_case_cauchy_problem} this is not
true. Assume $\phi$ is a weak solution to
\eqref{eq:diagonal_case_cauchy_problem}, then $\tilde{\phi} = \phi - g$ is a solution to
\begin{equation} \label{eq:weak_solution_pde_4}
	\left\{
		\begin{aligned}
			L \tilde{\phi} &= 0 & &\text{in } \Omega \\
			\tilde{\phi} &= 0 	& &\text{on } \partial \Omega 
		\end{aligned}
	\right.
\end{equation}
since $L \tilde{\phi} = L \phi - L g = L \phi$.

\begin{remark}
	
\end{remark}

By theorem \ref{teo:weak_solution}, problem \eqref{eq:weak_solution_pde_4} has a
weak solution. Nonetheless, we cannot specify whether there is unique solution
or not due to dichotomy \ref{item:alpha}--\ref{item:beta}, which is known as the
Fredholm alternative.
