
\subsection{Analytical solution}

Unlike the diagonal case, in the Smith--Hutton problem there is no clear choice for a fluid velocity to compute Péclet's number. The average velocity of the flow field given by
\begin{equation}
	\overline{v} = 
	\dfrac{\displaystyle \iint_\Omega \norm{\vb{v}} \dd{x} \dd{y}}{\displaystyle \iint_\Omega 1 \dd{x} \dd{y}} = 
	\frac{1}{2 L^2}
	\iint_\Omega \left\{ 4 y^2 (1 - x^2)^2 + 4 x^2 (1 - y^2)^2 \right\} \dd{x} \dd{y}
\end{equation}
could be chosen as a representative velocity. However, it is unclear whether or not it is important for the behaviour of the solution. In view of the PDE from \eqref{eq:smith_hutton_cauchy_problem}, it seems that the solution shall be governed by the quotient $\rho / \Gamma$. Due to the nature of the problem, the velocity field and the mixed boundary conditions (both Dirichlet and Neumann)

\subsubsection{Analytical solution for $\rho / \Gamma = \infty$}

As in the diagonal case, when $\rho / \Gamma = \infty$ we have $\mathrm{Pe} = \infty$. Dividing the PDE by Péclet's number results in the following transport problem:
\begin{equation} \label{eq:smith_hutton_cauchy_problem_infinite_peclet} 
	\left\{
	\begin{aligned}
		&u(x,y) \pdv{\phi}{x} + v(x,y) \pdv{\phi}{y} = 0 &
		&\text{in } \Omega \\
		&\phi = g & 
		&\text{on } C_1 \cup C_2 \\
		&\frac{\partial \phi}{\partial y} = 0 & 
		&\text{on } C_3
	\end{aligned}
	\right.
\end{equation}
We shall follow the method of characteristics again in order to give a solution of problem \eqref{eq:smith_hutton_cauchy_problem_infinite_peclet}. However, we will be less rigorous and we will not check the boundary conditions nor the uniqueness. Let $I \subset \real$ be an open interval and let $h \colon I \subset \real \to \real^2$, $s \mapsto h(s) = (x(s), y(s))$ be a mapping satisfying the following:
\begin{equation} \label{eq:smith_hutton_cauchy_problem_infinite_peclet_characteristics_1} 
	\left\{
	\begin{aligned}
		x'(s) &= u(x(s), y(s)) = 2 y (1 - x^2) \\
		y'(s) &= v(x(s), y(s)) = - 2 x (1 - y^2)
	\end{aligned}
	\right.
\end{equation}
Notice that \eqref{eq:smith_hutton_cauchy_problem_infinite_peclet_characteristics_1} is the same ODE system as \eqref{eq:velocity_smith_hutton_odes}, which implies the curves satisfying \eqref{eq:smith_hutton_cauchy_problem_infinite_peclet_characteristics_1} are streamlines. Hereinafter, we will use the terms streamline and characteristic as synonyms. We claim we already know all the characteristics in $\Omega$. Indeed, we have the following facts:
\begin{enumerate}[label={(\roman*)}, topsep=0pt] \label{lab:smith_hutton_cauchy_problem_infinite_peclet_i}
	\item When $x = \pm 1$ we have $u(\pm 1, y) = 0$ for all $y \in (0, 1)$ (there is only $y$--component of velocity). \label{lab:smith_hutton_cauchy_problem_infinite_peclet_x}
	\item When $y = 1$ we have $v(x, 1) = 0$ for all $x \in (-1,1)$ (there is only $x$--component of velocity). \label{lab:smith_hutton_cauchy_problem_infinite_peclet_y}
	\item When $y = 0$ we have $u(x,0) = 0$ for all $x \in (-1,1)$ and the $y$--component vanishes nowhere expect at $(0,0)$, but this point does not belong to $\Omega$. \label{lab:smith_hutton_cauchy_problem_infinite_peclet_iv}
	\item Any two streamlines cannot cross each other. Each point in $\Omega$ belongs to a single streamline. \label{lab:smith_hutton_cauchy_problem_infinite_peclet_v}
	\item The velocity field is non--zero at each point of $\Omega$.
\end{enumerate}
Facts \ref{lab:smith_hutton_cauchy_problem_infinite_peclet_x}, \ref{lab:smith_hutton_cauchy_problem_infinite_peclet_y} and \ref{lab:smith_hutton_cauchy_problem_infinite_peclet_v} imply that characteristics cannot exit $\Omega$ through the curve $C_2 = \left( \{ -L \} \times (0,L) \right) \cup \left( [-L,L] \times \{ L \} \right) \cup \left( \{ L \} \times [0,L) \right)$. Fact \eqref{lab:smith_hutton_cauchy_problem_infinite_peclet_iv} means that the velocity field ``pushes upwards'' on the curve $(-1,1) \times \{ 0 \}$, so characteristics cannot exit $\Omega$ through it either. Finally, fact \ref{lab:smith_hutton_cauchy_problem_infinite_peclet_y} implies that any two streamlines cannot die at the same point. Because all streamlines are contained in $\Omega$ and through each point passes a single streamlines, all characteristics begin at some point in $(-1,0) \times \{ 0 \}$ and finish at some point in $(0,1) \times \{ 0 \}$. At $(0,0)$ there is no characteristic starting or ending since the velocity field vanishes there. As a result, the initial value problem for characteristics is given by \eqref{eq:velocity_smith_hutton_ivp}, whose existence and uniqueness of solution we have already proved. 

Until now, we have taken points in the segment $(-1,1) \times \{ 0 \}$ although these do not belong to $\Omega$. Now we shall formalize this. Since the characteristics are solutions to the initial value problem \eqref{eq:velocity_smith_hutton_ivp}, these must be at least $\mathcal{C}^1$ curves in $\real^2$, hence must be continuous functions that admit a continuous extension to the segment $(-1,1) \times \{ 0 \}$.

Finally we find the solution to the transport problem. Consider the function $\varphi = \phi \circ h \colon I \subset \real \to \real$. By the chain rule we have
\begin{multline}
	\frac{\dd}{\dd{s}} \varphi(s) = 
	\frac{\dd}{\dd{s}} \phi(x(s), y(s)) = 
	\frac{\partial \phi}{\partial x} (x(s), y(s)) \, x'(s) + 
	\frac{\partial \phi}{\partial y} (x(s), y(s)) \, y'(s) = \\ =
	u(x(s),y(s)) \frac{\partial \phi}{\partial x} + 
	v(x(s),y(s)) \frac{\partial \phi}{\partial y} = 0
\end{multline}
where the last equality is true because of the PDE. This means, as we already know, that the restriction of $\phi$ to the characteristics is a constant function. Recall that the value of $\phi$ at the boundary segment $(-1,0) \times \{ 0 \}$ is given by $g$. If $C \subset \Omega$ is a characteristic, then the value of $\phi$ at each point $(x,y) \in C$ must the value that $g$ takes at the point of $(-1,0) \cup \{ 0 \}$ that $C$ intersects. In order to give an explicit form of the solution, we shall define a new function. Let $p = (x,y) \in \Omega$ and let $C \subset \Omega \cup ((-1,0) \times \{ 0 \})$ be the only characteristic containing $p$. Let $q \in ((-1,0) \times \{ 0 \})$ be the only point contained in the intersection $C \cap ((-1,0) \times \{ 0 \})$. We define the following mapping:
\begin{equation}
	\begin{aligned}
		\alpha \colon \Omega \subset \real^2 &\longrightarrow \real^2 \\
		p &\longmapsto \alpha(p) = q
	\end{aligned}
\end{equation}
The mapping $\alpha$ takes any point in $\Omega$ and applies it into the only point contained in the intersection of the characteristic $C$ containing $p$ and the segment $(-1,0) \times \{ 0 \}$. Intuitively, two nearby points $p_1, p_2 \in \Omega$ are contained in two characteristics $C_1, C_2$ that are close, so the image of $p_1, p_2$ by $\alpha$ are two nearby points as well. It could be argued that $\alpha$ is also a $\mathcal{C}^1$ mapping. A possible solution to the problem \eqref{eq:smith_hutton_cauchy_problem_infinite_peclet} is given by
\begin{equation}
	\phi(x,y) = 
	\left\{
	\begin{aligned}
		&(g \circ \alpha)(x,y) & &\text{if} (x,y) \in \overline{\Omega} \setminus (C_1 \cup C_2) \\
		&g(x,y) & &\text{otherwise}
	\end{aligned}
	\right.
	\quad (x,y) \in \overline{\Omega}
\end{equation}

\subsubsection{General problem}
