
\section{Numerical resolution of linear systems} \label{ap:numerical_resolution_linear_systems}

In this section we present several iterative methods and the LU decomposition,
all of them used to solve linear systems. The reference is Matrix Computations
by Gene H. Golub \cite{golub2013matrix}.

Let $\mathcal{M}_{n \times n}(\real)$ be the space of $n \times n$ matrices with
real coefficients. We consider linear systems of the form $A x = b$, where $A
\in \mathcal{M}_{n \times n}(\real)$ is the system matrix with $\det(A) \neq 0$
and $b \in \real^n$ is the vector of independent terms. Recall the following
definitions:

\begin{definition*}
    Let $A = (a_{ij}) \in \mathcal{M}_{n \times n}(\real)$ be a matrix. Then:
    \begin{enumerate}[label={(\roman*)}, topsep=0pt]
        \item $A$ is symmetric if $a_{ij} = a_{ji}$ for all $1 \leq i, j \leq n$.
        \item $A$ is positive definite if $c^\top A c > 0$ for all $c \in \real^n \setminus \{ 0 \}$.
        \item $A$ is diagonally dominant if
        \begin{equation*}
            \abs{a_{ii}} \geq \sum_{\substack{j = 1, j \neq i}}^n \abs{a_{ij}}, 
            \quad 1 \leq i \leq n
        \end{equation*}
        and strictly diagonally dominant if the previous inequality is strict.
    \end{enumerate}
\end{definition*}

\subsection{Iterative methods}

Consider the linear system $A x = b$ with $A = (a_{ij}) \in \mathcal{M}_{n
\times n}(\real)$ a non--singular matrix, $b \in \real^n$ the vector of
independent terms and $x = A^{-1} b \in \real^n$ the solution. Iterative methods
for linear systems generate a sequence $\{ x^{(k)} \}_{k \geq 0} \subset
\real^n$ that ideally converge to the solution, \ie $\lim_{k \to \infty} x^{(k)}
= x$. 

\subsubsection{Jacobi's method}

Let $x^{(k)}$ be the current approximation of $x = A^{-1} b$ and assume $a_{ii}
\neq 0$ for all $1 \leq n \leq n$. The first idea for an iterative method is to
compute $x^{(k+1)}$ as
\begin{equation} \label{eq:jacobi_method}
    x_i^{(k+1)} = 
    \frac{1}{a_{ii}} \left( b_i - \sum_{j=1}^n a_{ij} x_j^{(k)} \right), 
    \quad 1 \leq i \leq n
\end{equation}
The method given by \eqref{eq:jacobi_method} is known as Jacobi's method. We
have the following theorem about its convergence.

\begin{theorem}
    If $A \in \mathcal{M}_{n \times n}(\real)$ is strictly diagonally dominant,
    the Jacobi's method converges to $x = A^{-1} b$. 
\end{theorem}
\begin{proof}
    See \cite{golub2013matrix}, Chap. 11, pg. 615.
\end{proof}

\subsubsection{Gauss--Seidel's method}

Notice that before prior to calculate $x_i^{(k+1)}$ in \eqref{eq:jacobi_method},
the components $x_1^{(k+1)}, \ldots, x_{i-1}^{(k+1)}$ have had to be computed.
Since these are already available once $x_i^{(k+1)}$ is being calculated, a
natural improvement to Jacobi's method is the following:
\begin{equation*}
    x_i^{(k+1)} = 
    \frac{1}{a_{ii}} \left( b_i - \sum_{j=1}^{i-1} a_{ij} x_j^{(k+1)} - \sum_{j=i+1}^{n} a_{ij} x_j^{(k)} \right), 
    \quad 1 \leq i \leq n
\end{equation*}

Recall that $A$ is a positive definite matrix if $c^\top A c > 0$ for all $c \in
\real^n \setminus \{ 0 \}$. We say that $A$ is a diagonally dominant matrix if
it satisfies
\begin{equation*}
    \abs{a_{ii}} \geq \sum_{\substack{j = 1 \\ j \neq i}}^n \abs{a_{ij}}, 
    \quad 1 \leq i \neq n
\end{equation*}
The following theorem gives us two sufficient conditions so that Gauss--Seidel's
method converges.

\begin{theorem}
    If $A$ is a positive definite symmetric matrix or it is strictly diagonally
    dominant by rows, then the Gauss--Seidel converges to the linear system
    solution.
\end{theorem}
\begin{proof}
    See \cite{golub2013matrix}, Chap. 11, pg. 615.
\end{proof}

\subsubsection{Relaxation method}

From Gauss--Seidel's method we have
\begin{equation*}
    x_i^{(k+1)} = 
    \frac{1}{a_{ii}} 
    \left( b_i - \sum_{j=1}^{i-1} a_{ij} x_j^{(k+1)} - \sum_{j=i+1}^{n} a_{ij} x_j^{(k)} \right) =
    x_i^{(k)} + 
    \frac{1}{a_{ii}} \left( b_i - \sum_{j=1}^{i-1} a_{ij} x_j^{(k+1)} - \sum_{j=i}^{n} a_{ij} x_j^{(k)} \right)
\end{equation*}
which means that $x_i^{(k+1)}$ is equal to $x_i^{(k)}$ plus a correction. The
correction can be multiplied by a constant $\omega$
\begin{equation} \label{eq:relaxation_method}
    x_i^{(k+1)} = 
    x_i^{(k)} + 
    \frac{\omega}{a_{ii}} \left( b_i - \sum_{j=1}^{i-1} a_{ij} x_j^{(k+1)} - \sum_{j=i}^{n} a_{ij} x_j^{(k)} \right),
    \quad 1 \leq i \leq n
\end{equation}
so as to accelerate the convergence. The constant $\omega$ is known as
relaxation constant and the method given by \eqref{eq:relaxation_method} is
known as relaxation method. In general, there are no results providing an
optimal $\omega$ to improve the convergence velocity.

\subsubsection{Stop criterion}

An iterative method must stop at some point once the iteration $x^{(k)}$ is
close enough to the solution $x$. The distance between $x^{(k)}$ and $x$ is
given by $\norm{x^{(k)} - x}$. There are many norms in $\real^n$, thus a natural
question is which one to use. The following theorem asserts that the election of
norm is not relevant:

\begin{theorem}
    Any two norms $\norm{\cdot}_a$ and $\norm{\cdot}_b$ on $\real^n$ are
    equivalent, \ie there exist constants $c, C > 0$ such that
    \begin{equation*}
        c \norm{x}_a \leq \norm{x}_b \leq C \norm{x}_b
    \end{equation*}
    for all $x \in \real^n$.
\end{theorem}

\noindent
Hence any two norms on $\real^n$ provide the same notion of distance. A common
choice for iterative methods due to its low computational cost is the supremum
norm, given by
\begin{equation*}
    \norm{x}_\infty = \max_{1 \leq i \leq n} \abs{x_i}
\end{equation*}
To halt the iteration, one normally controls the norm $\norm{x^{(k+1)} -
x^{(k)}}$. Given a constant $\delta > 0$ small enough so that the approximation
of $x$ is good, the iteration is stopped when $\norm{x^{(k+1)} - x^{(k)}} <
\delta$.


\subsection{LU decomposition}

Let $U$ be an upper triangular non--singular matrix, that is to say, a matrix of
the form
\begin{equation*}
	U = 
	\begin{pmatrix}
		u_{11} 	& u_{12} & \cdots & u_{1,n-1} & u_{1n} \\
		0		& u_{22} & \cdots & u_{2,n-1} & u_{2n} \\
		\vdots	& \vdots & \ddots & \vdots 	  & \vdots \\
		0 		& 0		 & \cdots & u_{n-1,n-1} & u_{n-1,n} \\
		0 	   	& 0		 & \cdots & 0 			& u_{nn}
	\end{pmatrix},
	\
	u_{ii} \neq 0 \quad 1 \leq i \leq n
\end{equation*}
or equivalently
\begin{equation*}
	U = (u_{ij})_{i, j= 1 \divisionsymbol n} = 
	\left\{
	\begin{aligned}
		&u_{ij} = 0 		& 	&\text{if } i > j \\
		&u_{ij} \neq 0 		& 	&\text{if } i = j \\
		&u_{ij} \in \real 	& 	&\text{otherwise} 
	\end{aligned}
	\right.
\end{equation*}
Let $b \in \real^n$. The the linear system $U x = b$ can be easily solved with
the backward substitution algorithm
\begin{equation*}
	x_{n-i} =
	\frac{1}{u_{n-i,n-i}}
	\left(
	b_{n-i} - \sum_{j=n-i+1}^{n} u_{n-i,j} \, x_j
	\right),
	\quad 0 \leq i \leq n-1
\end{equation*}

Let $L$ be a lower triangular non--singular matrix,
\begin{equation*}
	L = 
	\begin{pmatrix}
		\ell_{11} 		& 0 			& \cdots & 0 				& 0  \\
		\ell_{21} 		& \ell_{22} 	& \cdots & 0 				& 0 \\
		\vdots 			& \vdots 		& \ddots & \vdots 	  		& \vdots \\
		\ell_{n-1,1} 	& \ell_{n-1,2} 	& \cdots & \ell_{n-1,n-1} 	& 0 \\
		\ell_{n1} 		& \ell_{n,2} 	& \cdots & \ell_{n,n-1} 	& \ell_{nn}
	\end{pmatrix},
	\
	\ell_{ii} \neq 0 \quad 1 \leq i \leq n
\end{equation*}
or which is the same,
\begin{equation*}
	L = (\ell_{ij}) = 
	\left\{
	\begin{aligned}
		&\ell_{ij} = 0 			& 	&\text{if } i < j \\
		&\ell_{ij} \neq 0 		& 	&\text{if } i = j \\
		&\ell_{ij} \in \real 	& 	&\text{otherwise} 
	\end{aligned}
	\right.
\end{equation*}
The system $L x = b$ can be solved with the forward substitution algorithm
\begin{equation*}
	x_i = 
	\frac{1}{\ell_{ii}} 
	\left(
	b_i - \sum_{j=1}^{i-1} \ell_{ij} \, x_j
	\right),
	\quad
	1 \leq i \leq n
\end{equation*}

\begin{theorem}
	Let $A \in \mathcal{M}_{n \times n}(\real)$.
	\begin{enumerate}[label={(\arabic*)}, topsep=0pt]
		\item Assume that there exist a lower triangular matrix $L$ and an
		upper triangular matrix $U$ such that $A = L U$. Then $L$ and $U$ are
		unique.
        \item If the $k$--th principal minor of $A$ is non--null for all $1 \leq
        k \leq n$, that is to say, if
        \begin{equation*}
            \begin{vmatrix}
                a_{11} & \cdots & a_{1k} \\
                \vdots & \ddots & \vdots \\
                a_{k1} & \cdots & a_{kk}
            \end{vmatrix} \neq 0
        \end{equation*}
        then there exist $L$ and $U$ such that $A = L U$.
        \item If the Gaussian elimination can be carried out on $A$, then $A = L U$ where
        \begin{equation*}
			L = 
			\begin{pmatrix}
				1 				& 0 			& \cdots & 0 				& 0  \\
				\ell_{21} 		& 1 			& \cdots & 0 				& 0 \\
				\vdots 			& \vdots 		& \ddots & \vdots 	  		& \vdots \\
				\ell_{n-1,1} 	& \ell_{n-1,2} 	& \cdots & 1 				& 0 \\
				\ell_{n1} 		& \ell_{n,2} 	& \cdots & \ell_{n,n-1} 	& 1
			\end{pmatrix}
			\qquad			
			U = 
			\begin{pmatrix}
				u_{11} 	& u_{12} & \cdots & u_{1,n-1} & u_{1n} \\
				0		& u_{22} & \cdots & u_{2,n-1} & u_{2n} \\
				\vdots	& \vdots & \ddots & \vdots 	  & \vdots \\
				0 		& 0		 & \cdots & u_{n-1,n-1} & u_{n-1,n} \\
				0 	   	& 0		 & \cdots & 0 			& u_{nn}
			\end{pmatrix}	
		\end{equation*}
	\end{enumerate}
\end{theorem}

Certain matrices are not suitable for Gaussian elimination. In such cases it is
necessary to permute the columns of $A$ during the elimination. This may be
expressed with an invertible permutation matrix $P$, so that $A$ is decomposed
as $P A = L U$ or $A = P^{-1} L U$. Once $A$ is decomposed, the linear system $P
A x = b$ can be solved as two triangular systems with the previously seen
algorithms. Indeed, since $P A x = L U x = b$, the systems to be solved are $L y
= b$ and $U x = y$. 
