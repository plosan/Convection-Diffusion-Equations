
\section{Some results on Measure Theory} \label{ap:measure_theory}

In this appendix we gather two important theorems needed to justify some steps
in the derivation of conservation laws in section
\ref{sec:the_convection_diffusion_equations}. Despite these results are basic, a
previous study of real analysis is required in order to understand and prove
them. A good reference for the interested reader is Real and Complex Analysis of
Walter Rudin \cite{rudin1987real}. 

\subsection{Differentiation under the integral sign}

Differentiation under the integral sign allows us to compute the derivative of
an integral of a function of two parameters in a simple way. It is needed, for
instance, when the mass conservation law or the heat diffusion equation are
derived. 

Let $(X, \mathcal{A}, \mu)$ be a measure space and let $[a, b] \subset \real$.
Hereinafter we deal with functions $f \colon X \times [a,b] \rightarrow \real$,
where $t \in [a, b]$ is the parameter on which $f$ depends. We assume that
$f(\cdot, t)$ is a measurable function for each $t \in [a, b]$.

\begin{theorem}[Differentiation under the integral sign]
	\label{theo:differentiation_under_the_integral_sign} Let $F(t) = \int_X
	f(x,t) \dd{\mu}$. Assume that
	\begin{enumerate}[label={(\roman*)}, topsep=0pt]
		\item $f(x,t_0)$ is an integrable function for some $t_0 \in
		[a,b]$.
		\item $\dfrac{\partial f}{\partial t}(x,t)$ is defined for all
		$(x, t) \in X \times [a, b]$.
		\item There exists an integral function $g \colon X \rightarrow \real$
		such that $\abs{\dfrac{\partial f}{\partial t} (x,t)} \leq
		g(x)$ for all $(x, t) \in X \times [a, b]$.
	\end{enumerate}
	Then $F$ is a differentiable function and
	\[
		F'(t) = \frac{\dd}{\dd t} F(t) = \int_X \frac{\partial f}{\partial t}(x,t) \dd{\mu}
	\]
\end{theorem}

For the applications needed in this project, $X = \real^m$ with $1 \leq m \leq
3$, $\mathcal{A}$ is the Borel $\sigma$--algebra on $\real^m$ and $\mu$ is
Lebesgue's measure on $\real^m$, which for most of the ``natural'' sets of
$\mathcal{A}$ coincides with the usual notion of $m$--dimensional volume.

\subsection{Lebesgue's differentiation lemma}

A common way to derive a conservation law is to integrate some functions in a
control volume, then apply Differentiation under the integral sign to obtain an
integral equation and finally get to a differential equation using Lebesgue's
differentiation lemma. 

An intuitive way to understand and to motivate Lebesgue's differentiation lemma
is the following. Let $f \colon \real \to \real$ be a continuous function, let
$a \in \real$ be a fixed point and let $F(x) = \int_a^x f(y) \dd{y}$, which is a
differentiable function. Due to a corollary of the Fundamental Theorem of
Calculus, we have $F'(x) = f(x)$. Using the definition of derivative,
\[
	F'(x) = 
	\lim_{h \to 0} \frac{F(x + h) - F(x)}{h} = 
	\lim_{h \to 0} \frac{1}{h} \left\{ \int_a^{x + h} f(y) \dd{y} - \int_a^{x} f(y) \dd{y} \right\} = 
	\lim_{h \to 0} \frac{1}{h} \int_x^{x + h} f(y) \dd{y} = f(x)
\]
Notice that the integral is divided by the length of the interval $[x, x+h]$,
otherwise the limit would be zero. Lebesgue's lemma generalizes the previous
equality by considering functions $f \colon \real^n \to \real$ and integrating
them on open balls $B(x_0, r) = \{ x \in \real^n \mid \norm{x -
x_0} < r\}$. Furthermore, the integral is divided by the $n$--dimensional
volume of $B(x_0, r)$, which is denoted by $\abs{B(x_0, r)}$.

\begin{theorem}[Lebesgue's differentiation lemma \cite{evans1998pde}]
	\label{eq:lebesgue_differentiation_lemma} Let $f \colon \real^n \rightarrow
	\real$ be a locally integrable function.
	\begin{enumerate}[label={(\arabic*)}, topsep=0pt]
		\item Then for almost everywhere point $x_0 \in \real^n$,
		\[
			\frac{1}{\abs{B(x_0, r)}} \int_{B(x_0,r)} f(x) \dd{x} \to f(x_0) 
			\quad \text{as } r \to 0
		\]
		\item In fact, for almost everywhere point $x_0 \in \real^n$,
		\[
			\frac{1}{\abs{B(x_0, r)}} \int_{B(x_0,r)} \abs{f(x) - f(x_0)} \dd{x} \to 0 
			\quad \text{as } r \to 0
		\]		
	\end{enumerate}
\end{theorem}
