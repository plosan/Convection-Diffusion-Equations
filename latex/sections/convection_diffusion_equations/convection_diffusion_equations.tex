
\subsection{General convection--diffusion equation}

Let $\cv{} \subset \real^n$ be a bounded control volume which may depend on time. Let $\phi \colon \real^n \times I \to \real$, $(\vb{x}, t) \mapsto \phi(\vb{x}, t)$ be a magnitude of the fluid (such as the concentration of some chemical substance) per unit of mass, where $I \subset \real$ is a non--degenerate interval. Then the total ammount of $\phi$ in $\cvt$ is
\begin{equation}
	\Phi(t) = 
	\int_{\cvt} \rho(\vb{x}, t) \phi(\vb{x}, t) \dd{\vb{x}}
\end{equation}
and its variation over time is
\begin{equation} \label{eq:general_cde_variation_Phi}
	\dot{\Phi}(t) =
	\frac{\dd}{\dd t} \Phi(t) = 
	\frac{\dd}{\dd{t}} \int_{\cvt} \rho(\vb{x}, t) \phi(\vb{x}, t) \dd{\vb{x}}
\end{equation}
The variation of $\Phi$ is a consequence of two contributions: the flux of $\phi$ through the control surface $\cst$ and the generation/elimination of $\phi$ in $\cvt$ due to source terms. Let $\vb{f} \colon \real \times \cs{} \times I \rightarrow \real^n$, $(\phi, \vb{x}, t) \mapsto \vb{f}(\phi, \vb{x}, t)$ be the vector field which gives the flux of $\phi$ through $\cs{}$. Then the total ammount of $\phi$ flowing through $\cst$ is given by
\begin{equation}
	\mathscr{F}(t) = \int_{\cst} \vb{f}(\phi, \vb{x}, t) \vdot \vb{n} \dd{S}
\end{equation}
In order to find out which sign has $\mathscr{F}(t)$, we may assume for a moment that there are no source terms. If $\mathscr{F}(t) > 0$, then $\phi$ is exiting $\cvt$ and, as a result, $\dot{\Phi}(t) < 0$. Conversely, if $\mathscr{F}(t) < 0$, then $\dot{\Phi}(t) > 0$, therefore $\dot{\Phi}(t)$ and $F(t)$ have opposite signs. Now, let $\dot{s}_\phi \colon \rightarrow \real$ be te source term, which provides the ammount of $\phi$ generated/eliminated in $\cvt$ per unit of time. Then the total ammount of $\phi$ generated/eliminated in $\cvt$ is
\begin{equation}
	\mathscr{S}(t) = \int_{\cvt} \dot{s}_\phi (\phi, \vb{x}, t) \dd{\vb{x}}
\end{equation}
Assume that there is no flux of $\phi$ through $\cst$, that is to say, $\mathscr{F}(t) = 0$. If $\phi$ is generated in $\cvt$, then $\mathscr{S}(t) > 0$, which implies $\dot{\phi}(t) > 0$; whereas if $\mathscr{S}(t) < 0$ then $\dot{\phi}(t) > 0$, thus $\dot{\phi}(t)$ and $\mathscr{S}(t)$ have the same sign. Introducing these terms in \eqref{eq:general_cde_variation_Phi} leads to
\begin{equation} \label{eq:general_cde_1}
	\frac{\dd}{\dd{t}} \int_{\cvt} \rho(\vb{x}, t) \phi(\vb{x}, t) \dd{\vb{x}} = 
	- \int_{\cst} \vb{f}(\phi, \vb{x}, t) \vdot \vb{n} \dd{S} 
	+ \int_{\cvt} \dot{s}_\phi (\phi, \vb{x}, t) \dd{\vb{x}}	
\end{equation}
Hereinafter we shall become less formal by omitting on which variables depends each function. In order to relate the flux $\vb{f}$ and $\phi$, we need to apply some constitutive law. Fourier's law for heat conduction and Fick's law for concentration state that $\vb{f}$ depends linearly on the gradient of $\phi$ with respect to the spatial variables \cite{evans1998heat}, that is,
\begin{equation} \label{eq:general_cde_flux_term_gradient_phi}
	\vb{f} = 
	-\Gamma_\phi \grad_{\vb{x}} \phi = 
	-\Gamma_\phi
	\begin{pmatrix}
		\displaystyle\pdv{\phi}{x_1} & \cdots & \displaystyle\pdv{\phi}{x_n}
	\end{pmatrix}^T
\end{equation}
where $\Gamma_\phi$ is known as the diffusion coefficient. So as not to complicate the notation, we will write $\grad{\phi}$ in place of $\grad_{\vb{x}} \phi$. Recall that $\grad{\phi} \in \real^n$ gives the direction of maximum growth of $\phi(\cdot, t)$ (the time is fixed because the gradient is computed with respect to $\vb{x}$). The minus sign in \eqref{eq:general_cde_flux_term_gradient_phi} is the consequence of heat (concentration of a chemical) flowing from regions of higher to lower temperature (concentration) regions. With this in mind, equation \eqref{eq:general_cde_1} is rewritten as
\begin{equation} \label{eq:general_cde_2}
	\frac{\dd}{\dd{t}} \int_{\cvt} \rho \phi \dd{\vb{x}} = 
	\int_{\cst} \Gamma_\phi \grad{\phi} \vdot \vb{n} \dd{S} +
	\int_{\cvt} \dot{s}_\phi \dd{\vb{x}}	
\end{equation}
and applying Reynolds transport theorem on the left--hand side of \eqref{eq:general_cde_2} with $\vb{b} = \vb{v}$,
\begin{equation}
	\int_{\cvt} \pdv{(\rho \phi)}{t} \dd{\vb{x}} + 
	\int_{\cst} \rho \phi \vb{v} \vdot \vb{n} \dd{S} = 
	\int_{\cst} \Gamma_\phi \grad{\phi} \vdot \vb{n} \dd{S} +
	\int_{\cvt} \dot{s}_\phi \dd{\vb{x}}	
\end{equation}
To turn surface integrals into volume integrals we apply divergence theorem,
\begin{equation}
	\int_{\cvt} \pdv{(\rho \phi)}{t} \dd{\vb{x}} + 
	\int_{\cvt} \div(\rho \phi \vb{v}) \dd{\vb{x}} = 
	\int_{\cvt} \div(\Gamma_\phi \grad{\phi}) \dd{\vb{x}} + 
	\int_{\cvt} \dot{s}_\phi \dd{\vb{x}}	
\end{equation}
Proceeding in a similar way to the continuity equation, we use Lebesgue's differentiation lemma in a ball $B(\vb{x}_0, r) \subset \cvt$ to obtain the general convection diffusion equation:
\begin{equation} \label{eq:general_cde_3}
	\pdv{(\rho \phi)}{t} + \div(\rho \phi \vb{v}) = 
	\div(\Gamma_\phi \grad{\phi}) + \dot{s}_\phi	
\end{equation}
The left--hand side of \eqref{eq:general_cde_3} can be expanded to find
\begin{equation} \label{eq:general_cde_4}
	\phi \left\{ \pdv{\rho}{t} + \div(\rho \vb{v}) \right\} + 
	\rho \pdv{\phi}{t} + \rho \vb{v} \cdot \grad{\phi} = 
	\div(\Gamma_\phi \grad{\phi}) + \dot{s}_\phi	
\end{equation}
Since the term between keys is the continuity equation, \eqref{eq:general_cde_4} is simplified to
\begin{equation} \label{eq:general_cde_general_convection_diffusion_equation}
	\rho \pdv{\phi}{t} + \rho \vb{v} \cdot \grad{\phi} = 
	\div(\Gamma_\phi \grad{\phi}) + \dot{s}_\phi	
\end{equation}
Equations \eqref{eq:general_cde_3} and \eqref{eq:general_cde_general_convection_diffusion_equation} are two equivalent forms of the same equation, each having its applications and benefits.




