
\section{Ordinary Differential Equations}

In this appendix we present a central theorem in basic Ordinary differential
equations (ODE) theory regarding the existence and uniqueness of solution to
initial value problems involving ODEs. Later this theorem is particularized to
the case of linear ODEs. Finally it is shown that any linear ODE of order $n$
can be rewritten as a linear system of ODEs.

Recall that an ordinary differential equation is an equation
\begin{equation*}
	g(t, x(t), x'(t), \ldots, x^{(n)}(t)) = 0
\end{equation*}
where the unknown $x(t) = (x_1(t), \ldots, x_m(t))^\top$ is a function of $m$
components and a variable $t \in \real$, $x' = \dfrac{\dd{x}}{\dd{t}}$ and $g(t,
y_1, \ldots, y_{n+1})$ with $y_1, \ldots, y_{n+1} \in \real^m$ is a function of
$1 + m(n+1)$ variables.

\subsection{General theory}

To start, we shall present the general initial value problem for first order
equations and a result which gives a formula to solve it. Let $U \subset \real
\times \real^n$ be an open set and $f \colon U \rightarrow \real^n$ a function.
Given $(t_0, x_0) \in U$ consider the initial value problem
\begin{equation} \label{eq:general_ode_ivp}
	\left\{
		\begin{aligned}
			\dot{x} &= f(t,x) \\
			x(t_0) &= x_0
		\end{aligned}
	\right.
\end{equation}
\noindent
The following proposition due to \cite{pau2020ode}, gives a formula to solve
problem \eqref{eq:general_ode_ivp}:
\begin{prop} \label{eq:ode_general_formula}
	Let $U \subset \real \times \real^n$ be an open set, $f \colon U \rightarrow
	\real^n$ a continuous function. Let $I \subset \real$ be an open interval
	and $x_0 \in \real^n$ such that $(t_0, x_0) \in U$. A function $x \colon I
	\rightarrow \real^n$ is a solution to \eqref{eq:general_ode_ivp} if and only
	if it is continuous and satisfies
	\begin{equation*}
		x(t) = x_0 + \int_{t_0}^t f(s,x(s)) \dd{s}, \quad t \in I
	\end{equation*}
\end{prop}

Now we can begin with the existence and uniqueness theorem. 
\begin{definition*}
	A function $f \colon \Omega \subset \real^n \rightarrow \real^m$ is said to
	be Lipschitz or Lipschitz continuous if there exists a constant $L$ such
	that
	\begin{equation*}
		\norm{f(x) - f(y)} \leq L \norm{x - y}
	\end{equation*}
	for every $x, y \in \Omega$. The constant $L$ is called the
	Lipschitz constant of $f$ \cite{salsa2009pde}.
\end{definition*}

Every Lipschitz function on $\Omega$ is also a continuous function in the usual
sense on $\Omega$. The converse is not true in general, and depends on $\Omega$.
The Lipschitz continuity is actually a very restrictive condition, since it
imposes that the function can grow at most as a linear function. 

\begin{definition*}
	Let $(E, \norm{\cdot}_E)$ and $(F, \norm{\cdot}_F)$ be two finite
	dimensional normed vector spaces and let $A \colon E \rightarrow F$ be a
	linear mapping between them. The norm of $A$ is defined to be
	\begin{equation*}
		\norm{A} \coloneqq 
		\sup{\left\{ \norm{A x}_F \mid x \in E, \ \norm{x}_E \leq 1 \right\}}
	\end{equation*}
\end{definition*}

\noindent
It is a well known fact that $\norm{A}$ is finite. In some cases, proving that a
function is Lipschitz continuous is a laborious task. In these situations we
have the following theorems:

\begin{theorem}[Corollary of the mean value theorem \cite{mazon2008calculo}]
	\label{teo:corollary_mean_value_theorem} Let $\Omega \subset \real^n$ an
	open set and let $f \colon \Omega \subset \real^n \to \real^m$ a
	differentiable function on $\Omega$. Let $x, y \in \Omega$ such
	that the segment $\overline{xy} = \{ \lambda x + (1 -
	\lambda) y \mid \lambda \in [0, 1] \}$ is contained in $\Omega$. Then
	the following inequality holds:
	\begin{equation*}
		\norm{f(x) - f(y)} \leq \sup_{z \in 
		\overline{xy}} \norm{D f(z)} \norm{x - y}
	\end{equation*}
\end{theorem}
\begin{proof}
	See \cite{mazon2008calculo}, page 78.
\end{proof}

\begin{theorem} \label{teo:c1_function_implies_lipschitz} Let $\Omega \subset
	\real^n$ a compact convex set and let $f \equiv (f_1, \ldots, f_m) \colon
	\Omega \subset \real^n \to \real^m$ a $\mathcal{C}^1(\Omega)$ function. Then
	$f$ is Lipschitz in $\Omega$.
\end{theorem}
\begin{proof}
	The differential of $f$, given by
	\begin{equation*}
		D f = 
		\begin{pmatrix}
			\pdv{f_1}{x_1} & \cdots & \pdv{f_1}{x_n} \\[6pt]
			\vdots & \ddots & \vdots \\[6pt]
			\pdv{f_m}{x_1} & \cdots & \pdv{f_m}{x_n} \\
		\end{pmatrix}
	\end{equation*}
	is a linear mapping $D f \colon \real^n \rightarrow \real^m$. Since $f$ is
	$\mathcal{C}^1(\Omega)$ function, the partial derivatives $\pdv{f_1}{x_1},
	\ldots, \pdv{f_m}{x_n}$ are continuous functions on $\Omega$. Moreover, the
	norm of the differential depends continuously on the partial derivatives
	$\pdv{f_1}{x_1}, \ldots, \pdv{f_m}{x_n}$ and on $z$. As a consequence,
	$\norm{D f}$ is a continuous function on $\Omega$. By Weierstrass theorem,
	$\norm{D f}$ reaches a maximum $M$ on $\Omega$. Applying theorem
	\ref{teo:corollary_mean_value_theorem} we have 
	\begin{equation*}
		\norm{f(x) - f(y)} \leq \sup_{z \in 
		\overline{xy}} \norm{D f(z)} \norm{x - y} \leq
		M \norm{x - y}
	\end{equation*}
	for all $x, y \in \Omega$, hence $f$ is Lipschitz on
	$\overline{\Omega}$.
\end{proof}

Finally we can state the Picard--Lindelöf theorem. Let $U$ be an open subset of
$\real^{n+1}$ and let $f \in C(U, \real)$, $(t_0, x_0) \in U$. Consider the
following IVP:
\begin{equation} \label{eq:appendix_ode_ivp}
	\left\{
	\begin{aligned}
		&\dot{x}(t) = f(t, x) \\
		&x(t_0) = x_0
	\end{aligned}
	\right.
\end{equation}
The Picard--Lindelöf theorem gives us the existence and uniqueness of solution
for \eqref{eq:appendix_ode_ivp}.

\begin{theorem}[Picard--Lindelöf \cite{teschl2012odes}]
	\label{teo:picard_lindelof} Suppose $f \in \mathcal{C}(U, \real^n)$, where
	$U$ is an open subset of $\real^{n+1}$, and $(t_0, x_0) \in U$. If $f$
	is locally Lipschitz continuous in the second argument, uniformly with
	respect to the first, then there exists a unique local solution $x(t)
	\in \mathcal{C}^1(I)$ of the initial value problem
	\eqref{eq:appendix_ode_ivp}, where $I$ is some interval around $t_0$. 
	
	More specifically, if $V = [t_0, t_0 + T] \times \overline{B(x_0,
	\delta)} \subset U$ and $M$ denotes the maximum of $\abs{f}$ on $V$, then
	the solution exists at least for $t \in [t_0, t_0 + T_0]$ and remains in
	$\overline{B(x_0, \delta)}$ where $T_0 = \min{\left\{ T,
	\frac{\delta}{M} \right\}}$. The analogous result holds for the interval
	$[t_0 - T, t_0]$.
\end{theorem}
\begin{proof}
	See \cite{teschl2012odes}, page 38.
\end{proof}


\subsection{Linear equations}

Let $\mathcal{M}_{n \times n}(\real)$ and $\mathcal{M}_{n \times n}(\cplex)$
denote the space of $n \times n$ matrices with real or complex coefficients,
respectively. Recall that an ODE or a system of ODEs is linear if it has the
form
\begin{equation} \label{eq:linear_ode}
	\dot{x} = A(t) x + b(t)
\end{equation}
where $A(t) \in \mathcal{M}_{n \times n}(\real)$ (or $A(t) \in \mathcal{M}_{n
\times n}(\cplex)$) and $b(t) \in \real^n$ for all $t \in I \subset \real$. The
following theorem establishes the existence and uniqueness of initial value
problems for linear ODEs.

\begin{theorem} \label{teo:linear_ode}
	Let $I \subset \real$ be an open interval and let $A \in \mathcal{C}(I,
	\mathcal{M}_{n \times n}(\real))$ and $b \in \mathcal{C}(I, \real^n)$. Then
	for all $(t_0, x_0) \in I \times \real^n$, the initial value problem
	\begin{equation*}
		\left\{
			\begin{aligned}
				&\dot{x} = A(t) x + b(t) & &t \in I \\
				&\dot{x}(t_0) = x_0
			\end{aligned}
		\right.
	\end{equation*}
	has a unique solution $\varphi \colon I \rightarrow \real^n$.
\end{theorem}

In order for theorem \ref{teo:linear_ode} to be useful for the project, we
shall show that any $n$--th order linear ODE
\begin{equation*}
	x^{(n)} + a_{n-1}(t) \, x^{(n-1)} + \cdots + a_1(t) \, x' + a_0(t) \, x = b(t) 
\end{equation*}
can be casted into a linear system of ODEs such as \eqref{eq:linear_ode}. We
define the functions
\begin{equation*}
	y_1 = x, \ y_2 = x', \ldots, \ y_{n-1} = x^{(n-2)}, \ y_n = x^{(n-1)}
\end{equation*}
thus
\begin{equation*}
	\left\{
		\begin{aligned}
			y_1' &= x' = y_2 \\
			y_2' &= x'' = y_3 \\
			&\vdots \\
			y_{n-1}' &= x^{(n-1)} = y_n \\
			y_n' &= x^{(n)} = -a_0(t) \, x - a_1(t) \, x' - \cdots - a_{n-1}(t) \, x^{(n-1)} + b(t) \\
			&= -a_0(t) \, y_1 - a_1(t) \, y_2 - \cdots - a_{n-1}(t) \, y_n + b(t)
		\end{aligned}
	\right.
\end{equation*}
which in matrix form is rewritten as:
\begin{equation} \label{teo:linear_ode_system}
	\begin{pmatrix}
		y_1' \\ y_2' \\ \vdots \\ y_{n-1}' \\ y_n'
	\end{pmatrix} = 
	\begin{pmatrix}
		0 & 1 & 0 & \cdots & 0 \\
		0 & 0 & 1 & \cdots & 0 \\
		\vdots & \vdots & \vdots & \ddots & \vdots \\
		0 & 0 & 0 & \cdots & 1 \\
		-a_0(t) & -a_1(t) & -a_2(t) & \cdots & a_{n-1}(t)
	\end{pmatrix}
	\begin{pmatrix}
		y_1 \\ y_2 \\ \vdots \\ y_{n-1} \\ y_n
	\end{pmatrix} + 
	\begin{pmatrix}
		0 \\ 0 \\ \vdots \\ 0 \\ b(t)
	\end{pmatrix}
\end{equation}
As long as $a_0, a_1, \ldots, a_{n-1}, b \colon I \subset \real \to \real$ are
continuous functions, theorem \ref{teo:linear_ode} can be applied to system
\eqref{teo:linear_ode_system}.


