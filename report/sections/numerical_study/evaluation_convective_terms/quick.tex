\subsubsection{Quadratic Upwind Interpolation for Convective Kinematics (QUICK)}

A logical improvement of CDS is using a parabola to interpolate between nodal
points rather than a straight line. To construct a parabola three points are
needed. As aforementioned, upstream conditions have a greater influence on flow
properties than downstream conditions for incompressible flows and low Mach
number gases. QUICK scheme takes profit of this fact. 

\begin{figure}[ht]
	\centering
	\begin{minipage}{.5\textwidth}
		\centering
		\begin{tikzpicture}
			% Points
			\def\zerox{0.5} \def\zeroy{1.5} \def\onex{3} \def\oney{2}
			\def\twox{5.5} \def\twoy{3} \def\coefa{1.5} \def\coefb{0.2}
			\def\coefc{0.04}
			% Ground
			\draw[thick] (0,0) -- ++(6,0);
			% Point W
			\filldraw[black] (\zerox,0) circle (2pt); \draw[dashed] (\zerox,0)
			-- ++(0,\zeroy); \node[black, yshift=-0.5cm] at (\zerox,0) {$W$};
			\node[black, yshift=-1cm] at (\zerox,0) {$(U)$}; \filldraw[sblue]
			(\zerox,\zeroy) circle (2pt); \node[blue, yshift=0.5cm] at
			(\zerox,\zeroy) {$\phi_W$};
			% Point P
			\filldraw[black] (\onex,0) circle (2pt); \draw[dashed] (\onex,0) --
			++(0,\oney); \node[black, yshift=-0.5cm] at (\onex,0) {$P$};
			\node[black, yshift=-1cm] at (\onex,0) {$(C)$}; \filldraw[sblue]
			(\onex,\oney) circle (2pt); \node[blue, yshift=0.5cm] at
			(\onex,\oney) {$\phi_P$};
			% Point e
			\filldraw[black] ({\onex+1},0) circle (2pt); \draw[dashed]
			({\onex+1},0) -- ++(0,2.34); \node[black, yshift=-0.5cm] at
			({\onex+1},0) {$e$}; \filldraw[sblue] ({\onex+1},2.34) circle (2pt);
			\node[blue, yshift=0.5cm] at ({\onex+1},2.34) {$\phi_e$};
			% Point E
			\filldraw[black] (\twox,0) circle (2pt); \draw[dashed] (\twox,0) --
			++(0,\twoy); \node[black, yshift=-0.5cm] at (\twox,0) {$E$};
			\node[black, yshift=-1cm] at (\twox,0) {$(D)$}; \filldraw[sblue]
			(\twox,\twoy) circle (2pt); \node[blue, yshift=0.5cm] at
			(\twox,\twoy) {$\phi_E$}; 
			% Blue line
			\draw[scale=1, domain=0:6, smooth, variable=\x, sblue, thick] plot
			({\x}, {\coefa + \coefb*(\x-\zerox) +
			\coefc*(\x-\zerox)*(\x-\onex)});
			% Mass flow
			\draw[-latex, red, thick] (3.25,0.75) -- ++(1.5,0)
			node[above]{$\dot{m}_e > 0$};
		\end{tikzpicture}
		\captionsetup{width=0.9\textwidth}
		\caption{QUICK when $(\vb{v} \vdot \vb{n})_e > 0$.}
		\label{fig:quick_1}
	\end{minipage}%
	\begin{minipage}{.5\textwidth}
		\centering
		\begin{tikzpicture}
			% Points
			\def\zerox{0.5} \def\zeroy{1.5} \def\onex{3} \def\oney{2.5}
			\def\twox{5.5} \def\twoy{3} \def\coefa{1.5} \def\coefb{0.4}
			\def\coefc{-0.04}
			% Ground
			\draw[thick] (0,0) -- ++(6,0);
			% Point P
			\filldraw[black] (\zerox,0) circle (2pt); \draw[dashed] (\zerox,0)
			-- ++(0,\zeroy); \node[black, yshift=-0.5cm] at (\zerox,0) {$P$};
			\node[black, yshift=-1cm] at (\zerox,0) {$(D)$}; \filldraw[sblue]
			(\zerox,\zeroy) circle (2pt); \node[blue, yshift=0.5cm] at
			(\zerox,\zeroy) {$\phi_P$};
			% Point e
			\filldraw[black] ({\zerox+1},0) circle (2pt); \draw[dashed]
			({\zerox+1},0) -- ++(0,1.96); \node[black, yshift=-0.5cm] at
			({\zerox+1},0) {$e$}; \filldraw[sblue] ({\zerox+1},1.96) circle
			(2pt); \node[blue, yshift=0.5cm] at ({\zerox+1},1.96) {$\phi_e$};
			% Point e
			\filldraw[black] (\onex,0) circle (2pt); \draw[dashed] (\onex,0) --
			++(0,\oney); \node[black, yshift=-0.5cm] at (\onex,0) {$E$};
			\node[black, yshift=-1cm] at (\onex,0) {$(C)$}; \filldraw[sblue]
			(\onex,\oney) circle (2pt); \node[blue, yshift=0.5cm] at
			(\onex,\oney) {$\phi_E$}; 
			% Point ee
			\filldraw[black] (\twox,0) circle (2pt); \draw[dashed] (\twox,0) --
			++(0,\twoy); \node[black, yshift=-0.5cm] at (\twox,0) {$EE$};
			\node[black, yshift=-1cm] at (\twox,0) {$(U)$}; \filldraw[sblue]
			(\twox,\twoy) circle (2pt); \node[blue, yshift=0.5cm] at
			(\twox,\twoy) {$\phi_{EE}$}; 
			% Blue line
			\draw[scale=1, domain=0:6, smooth, variable=\x, sblue, thick] plot
			({\x}, {\coefa + \coefb*(\x-\zerox) +
			\coefc*(\x-\zerox)*(\x-\onex)});
			% Mass flow
			\draw[-latex, red, thick] (0.75,0.75) -- ++(1.5,0)
			node[above]{$\dot{m}_e < 0$};
		\end{tikzpicture}
		\captionsetup{width=0.9\textwidth}
		\caption{QUICK when $(\vb{v} \vdot \vb{n})_e < 0$.}
		\label{fig:quick_2}
	\end{minipage}
\end{figure}

\noindent
Let $(x_0, \phi_0)$, $(x_1, \phi_1)$, $(x_2, \phi_2)$ be the points which the
polynomial $p(x)$ must interpolate, that is, $p(x_0) = \phi_0$, $p(x_1) =
\phi_1$ and $p(x_2) = \phi_2$, satisfying $x_0 < x_1 < x_2$. If $(\vb{v} \vdot
\vb{n})_e > 0$ then $x_0 = x_W$, $x_1 = x_P$ and $x_2 = x_E$, whereas $x_0 =
x_P$, $x_1 = x_E$ and $x_2 = x_{EE}$ in the case of $(\vb{v} \vdot \vb{n})_e <
0$. Let $p(x)$ be the following polynomial
\begin{equation*}
	p(x) = a_0 + a_1 (x - x_0) + a_2 (x - x_0) (x - x_1), \quad a_0, a_1, a_2 \in \real
\end{equation*}
Since the interpolating polynomial exists and is unique (see
\cite{quarteroni2010numerical8pinterp}, Theorem 8.1), by imposing the
interpolating condition, $p(x)$ will be the desired polynomial. The
interpolating condition is,
\begin{equation*}
	\left.
	\begin{aligned}
		p(x_0) &= a_0 = \phi_0 \\
		p(x_1) &= a_0 + a_1 (x_1 - x_0) = \phi_1 \\
		p(x_2) &= a_0 + a_1 (x_2 - x_0) + a_2 (x_2 - x_0) (x_2 - x_1) = \phi_2
	\end{aligned}	
	\right\}
\end{equation*}
which yields the following linear system:
\begin{equation*}
	\begin{pmatrix}
		1 & 0 & 0 \\
		1 & x_1 - x_0 & 0 \\
		1 & x_2 - x_0 & (x_2 - x_1)(x_2 - x_0)
	\end{pmatrix}
	\begin{pmatrix}
		a_0 \\ a_1 \\ a_2
	\end{pmatrix} = 
	\begin{pmatrix}
		\phi_0 \\ \phi_1 \\ \phi_2
	\end{pmatrix}
\end{equation*}
The determinant of the system matrix is non-zero because the abscissae are
distinct, therefore the solution is given by
\begin{equation*}
	\left.
	\begin{aligned}
		a_0 &= \phi_0 \\
		a_1 &= \frac{\phi_1 - \phi_0}{x_1 - x_0} \\
		a_2 &= \frac{\phi_2 - \phi_0}{(x_2 - x_1)(x_2 - x_0)} - \frac{\phi_1 - \phi_0}{(x_2 - x_1)(x_1 - x_0)}
	\end{aligned}	
	\right\}
\end{equation*}
and the polynomial is
\begin{equation*}
	p(x) = 
	\phi_0 - 
	\frac{(x - x_2) (x - x_0)}{(x_2 - x_1)(x_1 - x_0)} (\phi_1 - \phi_0) + 
	\frac{(x - x_1)(x - x_0)}{(x_2 - x_1)(x_2 - x_0)} (\phi_2 - \phi_0)
\end{equation*}
In terms of the $DCU$ notation, we have the following:
\begin{equation*}
	p(x) = 
	\left\{
	\begin{aligned}		
		&\phi_U - 
		\frac{(x - x_D)(x - x_U)}{(x_D - x_C)(x_C - x_U)} (\phi_C - \phi_U) + 
		\frac{(x - x_C)(x - x_U)}{(x_D - x_C)(x_D - x_U)} (\phi_D - \phi_U)
		& &\text{if } (\vb{v} \vdot \vb{n})_e > 0 \\
		&\phi_D - 
		\frac{(x - x_U)(x - x_D)}{(x_U - x_C)(x_C - x_D)} (\phi_C - \phi_D) + 
		\frac{(x - x_C)(x - x_D)}{(x_U - x_C)(x_U - x_D)} (\phi_U - \phi_D)
		& &\text{if } (\vb{v} \vdot \vb{n})_e < 0
	\end{aligned}
	\right.
\end{equation*}
Assuming a uniform grid, \ie $x_1 - x_0 = x_2 - x_1 = L$ and the face $f$
located at the midpoint between nodal points, the approximation of $\phi_e$
given by QUICK scheme is
\begin{equation*}
	\phi_e = -\frac{1}{8} \phi_0 + \frac{6}{8} \phi_1 + \frac{3}{8} \phi_2
\end{equation*}
and depending on the sign of $(\vb{v} \vdot \vb{n})_e$,
\begin{equation} \label{eq:quick_approximation}
	\phi_e = 
	\left\{
	\begin{aligned}
		&-\frac{1}{8} \phi_U + \frac{6}{8} \phi_C + \frac{3}{8} \phi_D & 
		&\text{if } (\vb{v} \vdot \vb{n})_e > 0 \\
		&-\frac{1}{8} \phi_D + \frac{6}{8} \phi_C + \frac{3}{8} \phi_U & 
		&\text{if } (\vb{v} \vdot \vb{n})_e < 0 \\
	\end{aligned}
	\right.
\end{equation}
The output \eqref{eq:quick_approximation} provided by QUICK scheme is
second--order accurate.