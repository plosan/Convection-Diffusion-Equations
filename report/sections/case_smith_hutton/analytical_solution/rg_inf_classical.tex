\subsubsection{Classical solution for \texorpdfstring{$\rho / \Gamma =
\infty$}{infinite rho/Gamma quotient}}

We assume $L = 1 \ \meter$. As in the diagonal case, when $\rho / \Gamma =
\infty$ we have $\peclet = \infty$. Dividing the PDE by Péclet's number results
in the following transport problem:
\begin{equation} \label{eq:smith_hutton_cauchy_problem_infinite_peclet_0} 
	\left\{
	\begin{aligned}
		u(x,y) \phi_x + v(x,y) \phi_y &= 0 	& &\text{in } \Omega \\
		\phi &= g 							& &\text{on } C_1 \cup C_2 \\
		\phi_y &= 0 						& &\text{on } C_3
	\end{aligned}
	\right.
\end{equation}
However, problem \eqref{eq:smith_hutton_cauchy_problem_infinite_peclet_0} is
overdetermined due to Neumann's boundary condition on $C_3$, thus we shall consider problem
\begin{equation} \label{eq:smith_hutton_cauchy_problem_infinite_peclet} 
	\left\{
	\begin{aligned}
		u(x,y) \phi_x + v(x,y) \phi_y &= 0 	& &\text{in } \Omega \\
		\phi &= g 							& &\text{on } C_1 \cup C_2
	\end{aligned}
	\right.
\end{equation}

\begin{definition*}
	A classical solution to problem
	\eqref{eq:smith_hutton_cauchy_problem_infinite_peclet} is a function $\phi
	\colon \overline{\Omega} \rightarrow \real$ such that:
	\begin{enumeratedef}
		\item $\phi \in \mathcal{C}^1(\overline{\Omega})$, that is to say,
		$\phi$ is a $\mathcal{C}^1(\overline{\Omega})$ function that admits a
		$\mathcal{C}^1$ extension to an open neighbourhood of each point of $\partial \Omega$,
		\item $\phi$ satisfies the PDE and
		\item $\phi$ satisfies the boundary conditions.
	\end{enumeratedef}
\end{definition*}

\noindent
In order to give a solution to
\eqref{eq:smith_hutton_cauchy_problem_infinite_peclet} we will follow the method
of characteristics again. 

\subsubsection*{Characteristic curves}

To begin we shall figure out how all the characteristic curves are. Take $(x_0,
y_0) \in \overline{\Omega}$ and consider a $\mathcal{C}^1$ mapping $\alpha(s) =
(x(s), y(s))$ such that
\begin{equation} \label{eq:smith_hutton_characteristics}
	\left\{
		\begin{aligned}
			x' &= u(\alpha(s)) = 2 y (1 - x^2) 		& &x(0) = x_0 \\
			y' &= v(\alpha(s)) = - 2 x (1 - y^2) 	& &y(0) = y_0 \\
		\end{aligned}
	\right.
\end{equation}
Consider the composition $h = \phi \circ \alpha$, which is well defined as long
as $\alpha(s) \in \Omega$. Then
\begin{multline}
	\frac{\dd}{\dd{s}} h(s) = 
	\frac{\dd}{\dd{s}} \phi(x(s), y(s)) = 
	\phi_x(x(s), y(s)) x'(s) + \phi_y(x(s), y(s)) y'(s) \\ =
	\phi_x(x(s), y(s)) u(x(s), y(s)) + 
	\phi_y(x(s), y(s)) v(x(s), y(s)) = 0 \label{eq:smith_hutton_derivative_composition}
\end{multline}
thus $\phi$ is constant on the curve given by $\alpha$, which means the curves
solution to \eqref{eq:smith_hutton_characteristics} are the characteristics for
problem \eqref{eq:smith_hutton_cauchy_problem_infinite_peclet}. Problems
\eqref{eq:vel_smith_hutton_ivp} and \eqref{eq:smith_hutton_characteristics} are
the same, hence the characteristics and streamlines are the same curves.
Applying proposition \eqref{prop:existence_uniqueness_streamlines}, the
characteristic passing through $(x_0, y_0) \in \overline{\Omega}$ exists and is
unique. In order to distinguish characteristics, we shall denote them by
$\alpha(s;x_0,y_0)$. 

Figure \ref{fig:smith_hutton_N201_streamlines} gives us the appearance of the
characteristics. Intuitively a characteristic starting at a point in the segment
$(-1,0) \times \{0\}$ should end at some point belonging to $(0,1) \times
\{0\}$. Nonetheless we must prove this. Before starting with the characteristics
in $\Omega$, we shall study the simpler characteristics in $\partial \Omega$.
The following proposition characterizes them:

\begin{prop} \label{prop:characteristics_1}
	Consider the initial value problem \eqref{eq:smith_hutton_characteristics}.
	\begin{enumerateprop}
		\item If $(x_0, y_0) \in \{ (0,0), \, (-1,1), \, (1,1) \}$, then
		$\alpha$ is a constant mapping.
		\item If $x_0 = -1$ and $y_0 \in [0, 1)$, then the image of $\alpha$ is
		the segment $\{ - 1 \} \times [0, 1)$ and $y$ is a
		strictly increasing function.
		\item If $x_0 = 1$ and  $y_0 \in [0,1)$, then the image of $\alpha$ is
		the segment $\{ 1 \} \times [0, 1)$ and $y$ is a strictly
		decreasing function.
		\item If $x_0 \in (-1,1)$ and $y_0 = 1$, then the image of $\alpha$ is
		the segment $(-1,1) \times \{ 1 \}$ and $x$ is a strictly increasing function.
	\end{enumerateprop}
\end{prop}
\begin{proof}
	We prove each point separately:
	\begin{enumerateprop}
		\item For  $(x_0, y_0) \in \{ (0,0), \, (-1,1), \, (1,1) \}$, we have $u
		= v = 0$, hence \eqref{eq:smith_hutton_characteristics} is rewritten as
		\[
			\left\{
				\begin{aligned}
					x' &= 0 	& &x(0) = x_0 \\
					y' &= 0 	& &y(0) = y_0 \\
				\end{aligned}
			\right.
		\]
		whose solution exists, is unique and is given by $\alpha(s) = (x_0,
		y_0)$ for all $s \in \real$.
		\item For $x_0 = -1$ we have $u = 0$ and $v(-1,y) = 2(1 - y^2)$ for all
		$y \in [0,1)$, therefore problem \eqref{eq:smith_hutton_characteristics}
		is
		\[
			\left\{
				\begin{aligned}
					x' &= 0 			& &x(0) = -1 \\
					y' &= 2(1 - y^2) 	& &y(0) = y_0 \\
				\end{aligned}
			\right.
		\]
		Since $y' > 0$ for $y_0 \in [0,1)$, we deduce $y$ is an strictly
		increasing function. Moreover $y_0 \in [0,1)$ is arbitray, thus the
		characteristic fills the segment $\{-1\} \times [0,1)$. The
		characteristic cannot reach $(-1,1)$ due to the uniqueness of solution.

		\item For $x_0 = 1$ we have $u = 0$ and $v(1,y) = -2(1 - y^2)$ for all
		$y \in [0,1)$, thus problem \eqref{eq:smith_hutton_characteristics} is
		\[
			\left\{
				\begin{aligned}
					x' &= 0 			& &x(0) = 1 \\
					y' &= -2(1 - y^2) 	& &y(0) = y_0 \\
				\end{aligned}
			\right.
		\]
		We have $y' < 0$ whenever $y_0 \in [0,1)$, so $y$ is a strictly
		decreasing function. As $y_0 \in [0,1)$ is arbitrary, the characteristic
		fills $\{ 1 \} \times [0,1)$, but it cannot reach $(1,1)$.

		\item If $y_0 = 1$ then $v(x,1) = 0$ and $u(x,1) = 2(1-x^2)$ for all
		$x_0 \in (-1,1)$. Problem \eqref{eq:smith_hutton_characteristics} is
		\[
			\left\{
				\begin{aligned}
					x' &= 2(1-x^2) 	& &x(0) = x_0 \\
					y' &= 0 		& &y(0) = 1 \\
				\end{aligned}
			\right.
		\]
		Function $x(s)$ is strictly increasing because $x' > 0$ and, since $x_0 \in
		(-1,1)$ is arbitrary, hence the image of the characteristic is $(-1,1) \times
		\{ 1 \}$.				
	\end{enumerateprop}
\end{proof}

Consider the set
\[
	\Omega_1 = ((-1,1) \times (0,1)) \setminus \{ (0,0) \} \subset \overline{\Omega}
\]
Proposition \eqref{prop:characteristics_1} gives how the characteristics with
initial condition $(x_0,y_0) \in \overline{\Omega} \setminus \Omega_1$ are. Now
we shall study the ones with $(x_0,y_0) \in \Omega_1$. It will be convenient to
define the function
\begin{align*}
	\tilde{f} \colon \real \times \real^2 &\longrightarrow \real^2 \\
	(s,x,y) &\longmapsto \tilde{f}(s,x,y) = 
	\begin{pmatrix}
		u(x,y) \\ v(x,y)
	\end{pmatrix}
\end{align*}
so that problem \eqref{eq:smith_hutton_characteristics} can be expressed as
\[
	\left\{
		\begin{aligned}
			\alpha' &= \tilde{f}(s,\alpha(s)) \\
			\alpha(0) &= (x_0,y_0)
		\end{aligned}	
	\right.	
\]
Notice that $\tilde{f}$ does not really depend on $s$ and $f(x,y) =
\tilde{f}(s,x,y)$, where $f$ is the function defined on
\eqref{eq:smith_hutton_velocity_mapping}. These two mappings will prove useful
now. 

\begin{prop} \label{prop:characteristics_extension} The solution
	$\alpha(s;x_0,y_0)$ with $(x_0,y_0) \in \Omega_1$ is defined in one of the
	following intervals
	\begin{enumerateprop}
		\item $I = [-m T, n T]$
		\item $I = [-m T, \infty)$ \label{item:interval2}
		\item $I = (-\infty, n T]$
		\item $I = \real$ \label{item:interval4}
	\end{enumerateprop}
	where
	\[
		T = \frac{3}{M}, \quad 
		M = \max{\left\{ \norm{f(x,y)} \mid (x,y) \in \overline{B((0,0), 5)} \right\}}
	\]
	for some natural numbers $m, n \geq 1$.
\end{prop}
\begin{proof}
	Let $T > 0$ be a constant to be determined and consider the sets $U = \real
	\times B((0,0), 5)$ and $V(x_0,y_0) = [-T,T] \times \overline{B((x_0,y_0),
	3)}$. The maximum of the distances between two points in $\overline{\Omega}$
	is $\sqrt{5}$, therefore
	\[
		\overline{\Omega} \subsetneq \overline{B((x_0,y_0), 3)} 
		\quad
		\text{and}
		\quad
		V(x_0,y_0) \subsetneq U
	\]
	for all $(x_0,y_0) \in \overline{\Omega}$. As $f \in \mathcal{C}(\real^2)$
	and $\tilde{f} \in \mathcal{C}(\real^3)$, we have $\norm{f} \in
	\mathcal{C}(\real^2)$ and $\norm{\tilde{f}} \in \mathcal{C}(\real^3)$.
	Moreover $f(x,y) = \tilde{f}(s,x,y)$ for all $s \in \real$ and $(x,y) \in
	\real^2$, so
	\[
		0 < 
		\sup_{V(x_0,y_0)} \norm{\tilde{f}} =
		\max_{V(x_0,y_0)} \norm{\tilde{f}} =
		\max_{\overline{B((x_0,y_0),3)}} \norm{f} \leq
		\max_{\overline{B((0,0),5)}} \norm{f} = M < 
		\infty
	\]
	and serves as upper bound for all $V(x_0,y_0)$. Now take $T = \frac{3}{M}$.
	By the Picard--Lindelöf theorem, we find that the solution exists for $s \in
	[-T, T]$ and stays in $\overline{B((x_0,y_0), 3)}$. Regarding the extension
	of the solution:
	\begin{itemize}[topsep=0pt]
		\item We can extend the solution for $s > 0$ with the same upper bound
		$M$ as long as $\alpha(T; x_0, y_0) \in
		\overline{\Omega}$\footnote{Actually, we can extend the solution as long
		as $(x_0,y_0) \in \overline{B((0,0),2)}$ since this ensures the solution
		will stay in $\overline{B((x_0,y_0), 5)}$ and therefore we can use the
		same bound $M$. However we are not interested on the characteristics
		when they exit $\overline{\Omega}$.}. Assume our solution is defined in
		$[-T, nT]$ for some natural number $n \geq 1$. If $\alpha(n T; x_0, y_0)
		\not\in \overline{\Omega}$, then the solution cannot be extended further
		with the same bound $M$. Otherwise, let $(x_n, y_n) = \alpha(n T; x_0,
		y_0) \in \overline{\Omega}$ and consider the problem
		\[
			\left\{
				\begin{aligned}
					\alpha' 	&= \tilde{f}(s,\alpha(s)) \\
					\alpha(n T) &= (x_n, y_n)
				\end{aligned}	
			\right.	
		\]
		Now take $V(x_n, y_n) = [-T,(n+1)T] \times \overline{B((\tilde{x}_0,
		\tilde{y}_0),3)}$. Following the same argument above, we find that the
		solution is defined for $s \in [-T, (n+1))T]$.
		\item We can extend the solution for $s < 0$ with the same upper bound
		$M$ as long as $\alpha(-T; x_0, y_0) \in \overline{\Omega}$. Following
		the same process as above, we find the solution is defined for $s \in
		[-(m+1)T, T]$.
	\end{itemize}
	\noindent
	If the process stops both for $s > 0$ and $s < 0$, the solution is defined
	in $I = \left[-\frac{3 m}{M}, -\frac{3 n}{M}\right]$ for some natural
	numbers $n, m \geq 1$. Otherwise, the solution is defined in an interval of
	the form \ref{item:interval2}--\ref{item:interval4}.
\end{proof}

\noindent
Finally the following proposition tells us that the characteristics with
$(x_0,y_0) \in \Omega_1$ eventually exit $\overline{\Omega}$.
\begin{prop}
	Consider the solution $\alpha(s;x_0,y_0)$ with $(x_0,y_0) \in \Omega_1$
	defined in an interval $I$ given by proposition 
	\begin{enumerateprop}
		\item There exists $T_1 \leq 0$ such that $\alpha(T_1;x_0,y_0) \in (-1,0)
		\times \{ 0 \}$.
		\item There exists $T_2 \geq 0$ such that $\alpha(T_2;x_0,y_0) \in (0,1)
		\times \{ 0 \}$.
	\end{enumerateprop}
\end{prop}
\begin{proof}
	Will be proved in a future version of this document. The idea is to prove
	that $\alpha(s;x_0,y_0)$ can be extended so that it is a maximal solution of
	problem \eqref{eq:smith_hutton_characteristics} defined on a maximal
	interval $(\omega_{-}, \omega_{+})$. The by means of a theorem in ODEs, we
	have that $\alpha(s;x_0,y_0) \to \partial B((0,0),5)$ when $s \to
	\omega_{\pm}$, hence at some point the solution must leave
	$\overline{\Omega}$. Since the curve
	\[
		\left( \{ -1 \} \times [0,1] \right) \cup 
		\left( (-1,1) \cup \{ 1 \} \right) \cup
		\left( \{ 1 \} \times [0,1] \right)
	\]
	is formed by the image of some characteristics according to proposition
	\ref{prop:characteristics_1}, the curve $\alpha(\cdot; x_0, y_0)$ must leave
	$\overline{\Omega}$ through the segment
	\[
		\left( (-1,1) \times \{ 0 \} \right) \setminus \{ (0,0) \}
	\]
	where $(u,v) \neq (0,0)$. We analyze the distinct cases:
	\begin{enumerate}[label={(\arabic*)}, topsep=0pt]
		\item If $(x_0,y_0) \in (-1,0) \times (0,1) \subset \Omega_1$,
		\begin{enumerate}[label={(1.\arabic*)}, topsep=0pt]
			\item when $s \to \omega_{-}$ (backward in time), we have $u, v <
			0$. Hence the solution must leave $\overline{\Omega}$ through
			$(-1,0) \times \{ 0 \}$, thus there exists a time $T_1$ such that
			$\alpha(T_1;x_0,y_0) \in (-1,0) \times \{ 0 \}$. \label{item:11}
			\item when $s \to \omega_{+}$ (forward in time), we have $u, v > 0$,
			therefore $\alpha(\cdot;x_0,y_0)$ eventually reaches $\{ 0 \} \times (0,1)$. \label{item:12}
		\end{enumerate}

		\item If $(x_0,y_0) \in \{ 0 \} \times (0,1) \Omega_1$,
		\begin{enumerate}[label={(2.\arabic*)}, topsep=0pt]
			\item when $s \to \omega_{-}$, $u, v < 0$
			and for some $s$ we have $\alpha(s) \in (-1,0) \times (0,1)$, which is case \ref{item:11}. \label{item:21}
			\item when $s \to \omega_{+}$, $u > 0, \, v < 0$
			and for some $s$ we have $\alpha(s) \in (0,1) \times (0,1)$, which is case \ref{item:32}.
		\end{enumerate}

		\item If $(x_0,y_0) \in (0,1) \times (0,1) \Omega_1$,
		\begin{enumerate}[label={(3.\arabic*)}, topsep=0pt]
			\item when $s \to \omega_{-}$, $u < 0, \, v > 0$
			and for some $s$ we have $\alpha(s) \in \{ 0 \} \times (0,1)$, which is case \ref{item:21}. \label{item:31}
			\item when $s \to \omega_{+}$, we have $u > 0, \,
			v < 0$. Hence the solution must leave $\overline{\Omega}$ through
			$(0,1) \times \{ 0 \}$, thus there exists a time $T_2$ such that
			$\alpha(T_2;x_0,y_0) \in (0,1) \times \{ 0 \}$.  \label{item:32}
		\end{enumerate}

	\end{enumerate}

\end{proof}

\subsubsection*{Solution}

As noted above, equation \eqref{eq:smith_hutton_derivative_composition} tells us
that $\phi$ is constant on the characteristic curve given by $\alpha(s;x,y)$ for
every $(x,y) \in \Omega_1$. Since the characteristic curve passing through
$(x,y) \in \Omega_1$ intersects the segment $(-1,0) \times \{ 0 \}$ at a point
$(\tilde{x}, \tilde{y})$ where the boundary condition $g$, the value is defined
of $\phi$ along the curve $\alpha(\cdot;x,y)$ is $g(\tilde{x}, \tilde{y})$. So
as to make this rigorous, given $(x,y) \in \Omega_1$, consider the mapping $h
\colon \Omega_1 \rightarrow \real$ such that $h(x,y) = T$ where $T$ satisfies
$\alpha(T;x,y) \in (-1,0) \times \{ 0 \}$. Observe that $h$ is well defined as a
result of the propositions in the previous section. Then the solution to
\eqref{eq:smith_hutton_cauchy_problem_infinite_peclet} is given by
\begin{equation} \label{eq:smith_hutton_infinite_peclet_solution}
	\phi(x,y) = 
	\left\{
		\begin{aligned}
			&g(\alpha(h(x,y);x,y)) 	& &\text{if } (x,y) \in \Omega_1					\\
			&g(x,y)					& &\text{if } (x,y) \in \Omega \setminus \Omega_1
		\end{aligned}
	\right.
\end{equation}

By construction, \eqref{eq:smith_hutton_infinite_peclet_solution} gives the
solution to \eqref{eq:smith_hutton_cauchy_problem_infinite_peclet}. However
proving that it is a classical solution is difficult since we do not have a
straightforward manner to know $h(x,y)$ nor $\alpha(h(x,y);x,y)$. The function
$g$ giving the boundary condition is of class $\mathcal{C}^1$. Intuitively the
mapping $\alpha(h(x,y);x,y)$ is also of class $\mathcal{C}^1(\Omega_1)$, but this
requires a rigorous proof. Notwithstanding, we can give an uniqueness result
based on formula \eqref{eq:smith_hutton_infinite_peclet_solution}.

\begin{theorem}
	If a classical solution to problem
	\eqref{eq:smith_hutton_cauchy_problem_infinite_peclet} exists, then it is unique.
\end{theorem}
\begin{proof}
	Let $\phi_1, \, \phi_2$ be two classical solutions to
	\eqref{eq:smith_hutton_cauchy_problem_infinite_peclet}. Consider the
	function $\phi = \phi_1 - \phi_2$, which is also of class
	$\mathcal{C}^1(\overline{\Omega})$. Since
	\begin{equation*}
		\begin{aligned}
			&u(x,y) (\phi_1)_x + v(x,y) (\phi_1)_y = 0 & &\text{in } \Omega \\
			&u(x,y) (\phi_2)_x + v(x,y) (\phi_2)_y = 0 & &\text{in } \Omega
		\end{aligned}
	\end{equation*}
	subtracting both equations yields 
	\begin{equation*}
		\begin{aligned}
			&u(x,y) \phi_x + v(x,y) \phi_y = 0 & &\text{in } \Omega
		\end{aligned}
	\end{equation*}
	that is to say, $\phi = \phi_1 - \phi_2$ solves the same PDE. Moreover
	$\phi_1$ and $\phi_2$ agree on $\partial \Omega$, hence $\phi = \phi_1 -
	\phi_2 = 0$ on $\partial \Omega$. With this in mind, $\phi$ satisfies the
	boundary--value problem 
	\begin{equation*}
		\left\{
		\begin{aligned}
			u(x,y) \phi_x + v(x,y) \phi_y &= 0 	& &\text{in } \Omega \\
			\phi &= 0 							& &\text{on } C_1 \cup C_2
		\end{aligned}
		\right.
	\end{equation*}
	whose solution, given by \eqref{eq:smith_hutton_infinite_peclet_solution},
	is $\phi \equiv 0$ on $\overline{\Omega}$. Therefore $\phi_1 = \phi_2$ and
	the classical solution is unique.
\end{proof}

% \noindent
% Finally the following proposition characterizes all the characteristic curves
% depending on $(x_0, y_0) \in \Omega$.

% \begin{prop}
% 	Consider the initial value problem \eqref{eq:smith_hutton_characteristics}, then:
% 	\begin{enumerateprop}
% 		\item If $(x_0, y_0) \in \{ (0,0), \, (-1,1), \, (1,1) \}$, then
% 		$\alpha$ is a constant mapping.
% 		\item If $x_0 = -1$ and $y_0 \in [0, 1)$, then the image of $\alpha$ is
% 		the curve $\{ - 1 \} \times [0, 1)$ and $y$ is a
% 		strictly increasing function.
% 		\item If $x_0 = 1$ and  $y_0 \in [0,1)$, then the image of $\alpha$ is
% 		the curve $\{ 1 \} \times [0, 1)$ and $y$ is a strictly
% 		decreasing function.
% 		\item If $x_0 \in (-1,1)$ and $y_0 = 1$, then the image of $\alpha$ is
% 		the curve $(-1,1) \times \{ 1 \}$ and $x$ is a strictly increasing function.
% 		\item If $(x_0,y_0) \in \Omega$, the characteristic can be extended so
% 		that the left end touches $(-1,0) \times \{ 0 \}$ and the right end touches $(0,1) \times \{ 0 \}$.
% 	\end{enumerateprop}
% \end{prop}
% \begin{proof}
% 	We will use extensively the following formula given by proposition \ref{eq:ode_general_formula}
% 	\begin{equation*}
% 		\alpha(s) = \alpha(0) + \int_0^s \alpha'(r) \dd{r}, \quad s \in I
% 	\end{equation*}
% 	or equivalently
% 	\begin{equation*}
% 		\left\{
% 			\begin{aligned}
% 				x(s) &= x(0) + \int_0^s x'(r) \dd{r} = 
% 				x(0) + \int_0^s u(\alpha(r)) \dd{r} \\
% 				y(s) &= y(0) + \int_0^s y'(r) \dd{r} = 
% 				x(0) + \int_0^s v(\alpha(r)) \dd{r}
% 			\end{aligned}
% 			\quad \text{for } s \in I \\
% 		\right.
% 	\end{equation*}
% 	where $I \subset \real$ is the interval given by proposition
% 	\ref{prop:characteristics_extension}. We prove each point separately:
% 	\begin{enumerateprop}
% 		\item At every point we have $u = v = 0$, therefore the unique solution
% 		to \eqref{eq:smith_hutton_characteristics} is $\alpha(s) = (x_0, y_0)$
% 		for all $s \in I$.
% 		\item We have $u = 0$ and $v > 0$ for each $y_0 \in [0,1)$ and $x_0 =
% 		-1$. Therefore $x(s) = -1$ 
% 	\end{enumerateprop}
% \end{proof}


% We shall follow the method of characteristics again in
% order to give a solution of problem
% \eqref{eq:smith_hutton_cauchy_problem_infinite_peclet}. However, we will be less
% rigorous and we will not check the boundary conditions nor the uniqueness. Let
% $I \subset \real$ be an open interval and let $h \colon I \subset \real \to
% \real^2$, $s \mapsto h(s) = (x(s), y(s))$ be a mapping satisfying the following:
% \begin{equation} \label{eq:smith_hutton_cauchy_problem_infinite_peclet_characteristics_1} 
% 	\left\{
% 	\begin{aligned}
% 		x'(s) &= u(x(s), y(s)) = 2 y (1 - x^2) \\
% 		y'(s) &= v(x(s), y(s)) = - 2 x (1 - y^2)
% 	\end{aligned}
% 	\right.
% \end{equation}
% Notice that
% \eqref{eq:smith_hutton_cauchy_problem_infinite_peclet_characteristics_1} is the
% same ODE system as \eqref{eq:velocity_smith_hutton_odes}, which implies the
% curves satisfying
% \eqref{eq:smith_hutton_cauchy_problem_infinite_peclet_characteristics_1} are
% streamlines. Hereinafter, we will use the terms streamline and characteristic as
% synonyms. We claim we already know all the characteristics in $\Omega$. Indeed,
% we have the following facts:
% \begin{enumerate}[label={(\roman*)}, topsep=0pt]
% \label{lab:smith_hutton_cauchy_problem_infinite_peclet_i}
% 	\item When $x = \pm 1$ we have $u(\pm 1, y) = 0$ for all $y \in (0, 1)$
% 	(there is only $y$--component of velocity).
% 	\label{lab:smith_hutton_cauchy_problem_infinite_peclet_x}
% 	\item When $y = 1$ we have $v(x, 1) = 0$ for all $x \in (-1,1)$ (there is
% 	only $x$--component of velocity).
% 	\label{lab:smith_hutton_cauchy_problem_infinite_peclet_y}
% 	\item When $y = 0$ we have $u(x,0) = 0$ for all $x \in (-1,1)$ and the
% 	$y$--component vanishes nowhere expect at $(0,0)$, but this point does not
% 	belong to $\Omega$.
% 	\label{lab:smith_hutton_cauchy_problem_infinite_peclet_iv}
% 	\item Any two streamlines cannot cross each other. Each point in $\Omega$
% 	belongs to a single streamline.
% 	\label{lab:smith_hutton_cauchy_problem_infinite_peclet_v}
% 	\item The velocity field is non--zero at each point of $\Omega$.
% \end{enumerate}
% Facts \ref{lab:smith_hutton_cauchy_problem_infinite_peclet_x},
% \ref{lab:smith_hutton_cauchy_problem_infinite_peclet_y} and
% \ref{lab:smith_hutton_cauchy_problem_infinite_peclet_v} imply that the
% characteristic curves cannot exit $\Omega$ through the curve
% $C_2 = \left( \{ -L \} \times (0,L) \right) \cup \left( [-L,L] \times \{ L \}
% \right) \cup \left( \{ L \} \times [0,L) \right)$. Fact
% \ref{lab:smith_hutton_cauchy_problem_infinite_peclet_iv} means that the
% velocity field ``pushes upwards'' on the curve $(-1,1) \times \{ 0 \}$, so
% characteristics cannot exit $\Omega$ through it either. Finally, fact
% \ref{lab:smith_hutton_cauchy_problem_infinite_peclet_y} implies that any two
% streamlines cannot die at the same point. Because all streamlines are contained
% in $\Omega$ and through each point passes a single streamlines, all
% characteristics begin at some point in $(-1,0) \times \{ 0 \}$ and finish at
% some point in $(0,1) \times \{ 0 \}$. At $(0,0)$ there is no characteristic
% starting or ending since the velocity field vanishes there. As a result, the
% initial value problem for characteristics is given by
% \eqref{eq:vel_smith_hutton_ivp}, whose existence and uniqueness of solution
% we have already proved. 

% Until now, we have taken points in the segment $(-1,1) \times \{ 0 \}$ although
% these do not belong to $\Omega$. Now we shall formalize this. Since the
% characteristics are solutions to the initial value problem
% \eqref{eq:vel_smith_hutton_ivp}, these must be at least $\mathcal{C}^1$
% curves in $\real^2$, hence must be continuous functions that admit a continuous
% extension to the segment $(-1,1) \times \{ 0 \}$.

% Finally we find the solution to the transport problem. Consider the function
% $\varphi = \phi \circ h \colon I \subset \real \to \real$. By the chain rule we
% have
% \begin{multline*}
% 	\frac{\dd}{\dd{s}} \varphi(s) = 
% 	\frac{\dd}{\dd{s}} \phi(x(s), y(s)) = 
% 	\frac{\partial \phi}{\partial x} (x(s), y(s)) \, x'(s) + 
% 	\frac{\partial \phi}{\partial y} (x(s), y(s)) \, y'(s) = \\ =
% 	u(x(s),y(s)) \frac{\partial \phi}{\partial x} + 
% 	v(x(s),y(s)) \frac{\partial \phi}{\partial y} = 0
% \end{multline*}
% where the last equality is true because of the PDE. This means, as we already
% know, that the restriction of $\phi$ to the characteristics is a constant
% function. Recall that the value of $\phi$ at the boundary segment $(-1,0) \times
% \{ 0 \}$ is given by $g$. If $C \subset \Omega$ is a characteristic, then the
% value of $\phi$ at each point $(x,y) \in C$ must the value that $g$ takes at the
% point of $(-1,0) \cup \{ 0 \}$ that $C$ intersects. In order to give an explicit
% form of the solution, we shall define a new function. Let $p = (x,y) \in \Omega$
% and let $C \subset \Omega \cup ((-1,0) \times \{ 0 \})$ be the only
% characteristic containing $p$. Let $q \in ((-1,0) \times \{ 0 \})$ be the only
% point contained in the intersection $C \cap ((-1,0) \times \{ 0 \})$. We define
% the following mapping:
% \begin{equation*}
% 	\begin{aligned}
% 		\alpha \colon \Omega \subset \real^2 &\longrightarrow \real^2 \\
% 		p &\longmapsto \alpha(p) = q
% 	\end{aligned}
% \end{equation*}
% The mapping $\alpha$ takes any point in $\Omega$ and applies it into the only
% point contained in the intersection of the characteristic $C$ containing $p$ and
% the segment $(-1,0) \times \{ 0 \}$. Intuitively, two nearby points $p_1, p_2
% \in \Omega$ are contained in two characteristics $C_1, C_2$ that are close, so
% the image of $p_1, p_2$ by $\alpha$ are two nearby points as well. It could be
% argued that $\alpha$ is also a $\mathcal{C}^1$ mapping. A possible solution to
% the problem \eqref{eq:smith_hutton_cauchy_problem_infinite_peclet} is given by
% \begin{equation*}
% 	\phi(x,y) = 
% 	\left\{
% 	\begin{aligned}
% 		&(g \circ \alpha)(x,y) & &\text{if} (x,y) \in \overline{\Omega} \setminus (C_1 \cup C_2) \\
% 		&g(x,y) & &\text{otherwise}
% 	\end{aligned}
% 	\right.
% 	\quad (x,y) \in \overline{\Omega}
% \end{equation*}

% \begin{theorem}
% 	If a classical solution exists, then it is unique.
% \end{theorem}